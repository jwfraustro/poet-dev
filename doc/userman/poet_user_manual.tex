\documentclass[letterpaper,12pt]{article}
%\usepackage{margin}
\usepackage{graphicx}
\usepackage{enumitem}

\usepackage{amssymb, amsmath}
%\usepackage{fancyvrb}
%\usepackage{fixltx2e}

\usepackage[usenames,dvipsnames]{xcolor}
%\usepackage{xcolor}

\usepackage{listings}
%\usepackage{pxfonts}
%\usepackage[tiny,compact]{titlesec}
%\usepackage{bera}
%\usepackage{alltt}
%\renewcommand{\ttdefault}{txtt}

% To use boldface verbatim:
%\lstset{basicstyle=\ttfamily,
%        escapeinside={||},
%        mathescape=true}

\lstset{
    language={[LaTeX]TeX},
    basicstyle=\tt\color{red},
    escapeinside={||},
}

%\bibliographystyle{jupiter}
\bibliographystyle{aa}
%\bibliographystyle{apj}
%\bibliographystyle{icarus}

\def\bibAnnoteFile#1{}
\usepackage{natbib}
\bibpunct[, ]{(}{)}{,}{a}{}{,}
\usepackage{astjnlabbrev-jh}
\usepackage{bibentry}
\setlength\bibsep{0pt}
\usepackage{commath}
\usepackage{rotating}


% :::: jhmacs2.tex :::::

\typeout{Joe Harrington's personal setup, Wed Jun 17 10:53:17 EDT 1998}
% :::::: pato.tex ::::::
% Joetex character unreservations.
% This file frees most of TeX's reserved characters, and provides
% several alternatives for their functions.

% Tue Mar 29 22:23:03 EST 1994

% utility
\catcode`@=11

% comments are first....

\long\def\comment#1{}
\def\com{}
%\def\commenton{\catcode`\%=14}
%\def\commentoff{\catcode`\%=12}

\comment{$}
\let\mathshift=$
\def\mathstart{\begingroup $}
\def\mathstop{$ \endgroup}
\def\math#1{$#1$}
\def\mathshifton{\catcode`\$=3}
\def\mathshiftoff{\catcode`\$=12}

\def\greek#1{\math{#1}}

\let\oldbackslash=\backslash
\def\backslash{\ifmmode\oldbackslash\else$\oldbackslash$\fi}

\comment{alignment tab}
\let\atab=&
\def\atabon{\catcode`\&=4}
\def\ataboff{\catcode`\&=12}

\comment{super- and subscripts -- treat as a unit. \sp and \sb already
exist in plain TeX math mode, but not in regular text.  In fact,
superscripting and subscripting are automatic only in math mode (see
TeXbook p. 134.  Elsewhere, they either need to be simulated or faked
in math mode.  Use math mode or \msb and \msb when dealing with
numbers or they get too big (compare 10\sb{2} to \math{10\sb{2}}).}
\let\oldmsp=\sp
\let\oldmsb=\sb
\def\sp#1{\ifmmode
	   \oldmsp{#1}%
	 \else\strut\raise.85ex\hbox{\scriptsize #1}\fi}
\def\sb#1{\ifmmode
	   \oldmsb{#1}%
	 \else\strut\raise-.54ex\hbox{\scriptsize #1}\fi}
\newbox\@sp
\newbox\@sb
\def\sbp#1#2{\ifmmode%
	   \oldmsb{#1}\oldmsp{#2}%
	 \else
	   \setbox\@sb=\hbox{\sb{#1}}%
	   \setbox\@sp=\hbox{\sp{#2}}%
	   \rlap{\copy\@sb}\copy\@sp
	   \ifdim \wd\@sb >\wd\@sp
	     \hskip -\wd\@sp \hskip \wd\@sb
	   \fi
	\fi}
\def\msp#1{\ifmmode
	   \oldmsp{#1}
	 \else \math{\oldmsp{#1}}\fi}
\def\msb#1{\ifmmode
	   \oldmsb{#1}
	 \else \math{\oldmsb{#1}}\fi}
\def\supon{\catcode`\^=7}
\def\supoff{\catcode`\^=12}
\def\subon{\catcode`\_=8}
\def\suboff{\catcode`\_=12}
\def\supsubon{\supon \subon}
\def\supsuboff{\supoff \suboff}


\comment{active character -- seems pointless to have a function to
replace the single-character-to-replace-a-function.  If you want ~ as
an active character, reenable it.}
\def\actcharon{\catcode`\~=13}
\def\actcharoff{\catcode`\~=12}


\comment{parameter -- just turn on and off when needed.  Doing the
following:

\let\defplain=\def
\defplain\def{\paramon\afterassignment\paramoff\defplain}

barfs in \long\def.

Putting \paramon \paramoff in \begin{table} and \end{table} might be
good... amgreene mentionned something about \halign, too.

}

\def\paramon{\catcode`\#=6}
\def\paramoff{\catcode`\#=12}

\comment{brackets -- now these produce the Spanish inverted
punctuation.  \spanishoff will return them to their normal English
function of being angle brackets.}

\let\spanishexcl=<
\let\spanishques=>
\let\oldlt=<
\let\oldgt=>
\def\spanishexclon{\catcode`\<=12}
\def\spanishexcloff{\catcode`\<=13
		\def <{\ifmmode\oldlt\else$\oldlt$\fi}}
\def\spanishqueson{\catcode`\>=12}
\def\spanishquesoff{\catcode`\>=13
		\def >{\ifmmode\oldgt\else$\oldgt$\fi}}

\def\spanishon{\spanishexclon \spanishqueson}
\def\spanishoff{\spanishexcloff \spanishquesoff}

\comment{ now fix some things this will break}

\comment{ ref
\let\oldref=\ref
\def\ref#1{\reservedcharson\oldref{#1}\reservedcharsoff}
}

\comment{ bibliography }
\let\oldcite=\cite
\def\cite#1{\reservedcharson\oldcite{#1}\reservedcharsoff}

\let\oldbibliography=\bibliography
\def\bibliography#1{\reservedcharson\oldbibliography{#1}\reservedcharsoff}

\comment{ And now to turn us totally on and off... }

\def\reservedcharson{ \mathshifton   \actcharon   \paramon}
\def\reservedcharsoff{\mathshiftoff  \actcharoff  \paramoff}

\comment{ this doesn't seem to work right for definitions...}
\def\nojoe#1{\reservedcharson#1\reservedcharsoff}

\catcode`@=12
\reservedcharson

% ::::::::::::::::::::::

\comment{joe.tex sets up \comment and character reservation stuff.
  Remember to turn on reserved characters before including other
  peoples' files, and before defining anything.}

\reservedcharson

\def\atachar{\catcode`@=11}
\def\atnotachar{\catcode`@=12}

\atachar

\comment{My macros.}
\comment{technical stuff}

\let\oldpm=\pm
\def\pm{\ifmmode\oldpm\else\math{\oldpm}\fi}
\def\by{\ifmmode\times\else\math{\times}\fi}
\newcommand\lt{\ifmmode<\else\math{<}\fi}
\newcommand\gt{\ifmmode>\else\math{>}\fi}
\let\oldsim=\sim
\def\sim{\ifmmode\oldsim\else\math{\oldsim}\fi}
\let\oldapprox=\approx
\def\approx{\ifmmode\oldapprox\else\math{\oldapprox}\fi}

\def\ttt#1{10\sp{#1}}
\def\tttt#1{\by\ttt{#1}}
\def\bttt#1{10\sp{{\bfseries #1}}}
\def\btttt#1{{\by\bttt{#1}}}

\comment{\input{greek.tex}}
\DeclareSymbolFont{UPM}{U}{eur}{m}{n}
\DeclareMathSymbol{\umu}{0}{UPM}{"16}
\let\oldumu=\umu
\renewcommand\umu{\ifmmode\oldumu\else\math{\oldumu}\fi}
\newcommand\micro{\umu}
\comment{\font\greek = psyr
\def\micro{{\greek m}}}
\comment{\def\micron{\math{\mu}m}}
\def\micron{\micro m}
\let\microns \micron
\def\microsec{\micro s}

\def\angstrom{\AA}
\let\angstroms \angstrom

\def\degree{\ifmmode\sp\circ\else\math{\sp{\circ}}\fi}
\let\degrees \degree
\def\decdegree{\ifmmode\rlap{.}\sp{\circ}\else\rlap{.}\math{\sp{\circ}}\fi}
\def\arcmin{\ifmmode\sp{'}\else\math{\sp{'}}\fi}
\def\decarcmin{\ifmmode\rlap{.}\sp{'}\else\rlap{.}\math{\sp{'}}\fi}
\def\arcsec{\ifmmode\sp{''}\else\math{\sp{''}}\fi}
\def\decarcsec{\ifmmode\rlap{.}\sp{''}\else\rlap{.}\math{\sp{''}}\fi}

\def\prim{\ifmmode\sp{\prime}\else\math{\sp{prime}}\fi}

\def\h{\sp{h}}
\def\m{\sp{m}}
\def\s{\sp{s}}
\def\decs{\rlap{.}\sp{s}}

\def\smfrac#1#2{\math{#1\over#2}}

\def\bm#1{{\mbox{\boldmath\math{#1}}}}

\comment{writing}

\comment{
\def\ie{{\em i.e.,}\ }
\def\eg{{\em e.g.,}\ }
}
\def\etal{{\em et al.}}

\comment{for aligning tables (\i was a dotless i, \ii is for ``naive'')}
\let\oi=\i
\def\ii{\"\oi}
\newbox{\wdbox}
\def\c{\setbox\wdbox=\hbox{,}\hspace{\wd\wdbox}}
\def\i{\setbox\wdbox=\hbox{i}\hspace{\wd\wdbox}}
\def\n{\hspace{0.5em}}
\def\m{\hspace{1em}}
\def\d{\phantom{$-$}}

\comment{good for addresses}
\def\tablebox#1{\begin{tabular}[t]{@{}l@{}}#1\end{tabular}}

\comment{notes}
\comment{\font\tinytt = cmtt8}
\def\marnote#1{\marginpar{\raggedright\tiny\ttfamily\baselineskip=9pt #1}}
\def\herenote#1{{\bfseries #1}\typeout{======================> note on page \arabic{page} <====================}}
\def\fillin{\herenote{fill in}}
\def\fillref{\herenote{ref}}

\comment{Page setup like I like it.}

\comment{ % margin settings -- change over to jhmar?
% \input{simplemargins.sty}
% \comment{paper isn't really 8.5x11 inches}
% \setlength{\smpageheight}{7.96875in}
% \setlength{\smpageheight}{10.96875in}
% \setallmargins{3.0cm}
}

\comment{% paragraphs
% \setlength{\parskip}{\baselineskip}
% \setlength{\parindent}{0em}
}

\comment{figs/tables}
\setcounter{totalnumber}{400}

\def\nocaption{\refstepcounter\@captype \@dblarg{\@nocaption\@captype}}

\long\def\@nocaption#1[#2]#3{\addcontentsline{\csname
  ext@#1\endcsname}{#1}{\protect\numberline{\csname 
  the#1\endcsname}{\ignorespaces #2}}{\ignorespaces #3}}

\comment{Date and time format}
\def\today{\number\day\space \ifcase\month\or
  January\or February\or March\or April\or May\or June\or
  July\or August\or September\or October\or November\or December\fi
  \space\number\year}

\newcount\@timect
\newcount\@hourct
\newcount\@minct
\def\now{\@timect=\time \divide\@timect by 60
	 \@hourct=\@timect \multiply\@hourct by 60
	 \@minct=\time \advance\@minct by -\@hourct
	 \number\@timect:\ifnum \@minct < 10 0\fi\number\@minct}

\comment{optional file inclusion}
\def\inclopt#1#2{\if#2y#1,\fi}
\def\texfopt#1#2{\if#2y#1\else null\fi}
\def\psfopt#1#2{\if#2y#1\else/dev/null\fi}

\comment{page styling}
\def\clearblankdoublepage{\clearpage\if@twoside \ifodd\c@page\else
    {\pagestyle{empty}\hbox{}\newpage\if@twocolumn\hbox{}\newpage\fi}\fi\fi}

\def\threehfbox#1#2#3{\hbox to \textwidth{\com
	\rlap{#1}\com
	\rlap{\hbox to \textwidth{\hfil #2 \hfil}}\com
	{\hfil #3}}}

\atnotachar
\reservedcharsoff

% ::::::::::::::::::::::


\def\vs{{\em vs.}}
\def\p{\phantom{(0)}}

\setcounter{secnumdepth}{3}
\actcharon
\renewcommand{\textfraction}{0.1}
\comment{\paramon\def\herenote#1{}\paramoff}
\renewcommand{\thepage}{\arabic{page}}
\reservedcharson
\comment{Must have ONLY ONE of these...}
\newcommand\jhauth[1]{{#1}}
\newcommand\jhstud[1]{{#1}}
\comment{
\newcommand\jhauth[1]{{\bfseries #1}}
\newcommand\jhstud[1]{{\em #1}}
}
% :::::::::::: My Additions ::::::::::::::
\newcommand\Spitzer{{\em Spitzer}}
\newcommand\SST{{\em Spitzer Space Telescope}}
\newcommand\chisq{$\chi^2$}
\newcommand\PT{$P$--$T$}
\newcommand\itbf[1]{\textit{\textbf{#1}}}
\newcommand\bftt[1]{\texttt{\textbf{#1}}}
\newcommand\function[1]{\noindent\texttt{\begin{tabular}{@{}l@{}l}#1\end{tabular}}\newline}
\newcommand\bfv[1]{|\textbf{#1}|}
\newcommand\ttred[1]{\textcolor{red}{\ttfamily #1}}
\newcommand\ttblue[1]{\textcolor{blue}{\ttfamily #1}}
\definecolor{gray}{gray}{0.6}
\newcommand\tgray[1]{\textcolor{gray}{#1}}

\newcommand\der{{\rm d}}
\newcommand\tno{$\sp{-1}$}
\newcommand\tnt{$\sp{-2}$}
\newcommand*\Eval[3]{\left.#1\right\rvert_{#2}^{#3}}
\newcommand\findme[1]{\herenote{(FINDME: #1)}}
%:::::::::::::::::::::::::::::::::::::::::
% Next six lines adjust spacing above/below captions and Sections etc
% Adjust as needed
% Python style for highlighting
\DeclareFixedFont{\ttnm}{T1}{txtt}{m}{n}{9.8}  % for normal

\newcommand\plainstyle{\lstset{
language=Python,
basicstyle = \ttnm,
keywordstyle  = \ttnm,      %
emph        = {MyClass, __init__},     % Custom highlighting
emphstyle   = \ttnm\color{black},    % Custom  highlighting style
stringstyle = \color{black},         % Strings highlighting style
commentstyle=\color{black},         % Comment highlighting style
frame       = tb,                      % Any extra options here
showstringspaces = false
}}

% Plain environment:
\lstnewenvironment{plain}[1][]{\plainstyle\lstset{#1}}{}

\newcommand\plaininline[1]{{\plainstyle\lstinline!#1!}}

\comment{
% \setlength{\abovecaptionskip}{0pt}
% \setlength{\belowcaptionskip}{0pt}
% \setlength{\textfloatsep}{8pt}
% \titlespacing{\section}{0pt}{5pt}{*0}
% \titlespacing{\subsection}{0pt}{5pt}{*0}
% \titlespacing{\subsubsection}{0pt}{5pt}{*0}
}

\reservedcharsoff
\actcharon
\mathshifton

\usepackage{epsfig}
%\usepackage{hyperref}
\textwidth=6.5in
\textheight=9.5in
\topmargin=-0.75in
\oddsidemargin=0.0in
\evensidemargin=0.0in

\pagestyle{myheadings}
\markright{POET}
\pagenumbering{arabic}
\graphicspath{fig}
\begin{document}

\begin{titlepage}
\begin{center}

\textsc{\LARGE University of Central Florida}\\[1.5cm]

% Title
\rule{\linewidth}{0.5mm} \\[0.4cm]
{ \huge \bfseries POET User Manual \\[0.4cm] }
\rule{\linewidth}{0.5mm} \\[1.0cm]

\textsc{\Large A Data Analysis Code for Transiting Exoplanets}\\[1.5cm]

% Author and supervisor
\noindent
\begin{minipage}{0.4\textwidth}
\begin{flushleft} \large
\emph{Authors:}\\
Ryan      \textsc{Challener} \\
Zacchaeus \textsc{Scheffer}\\ \vspace{1.2cm}
\end{flushleft}
\end{minipage}%
\begin{minipage}{0.4\textwidth}
\begin{flushright} \large
\emph{Supervisor:} \\
Dr.~Joseph \textsc{Harrington}\\ \vspace{1.2cm}
\end{flushright}
\end{minipage}

\vfill

% Bottom of the page
{\large \today}

\end{center}
\end{titlepage}


\section{Introduction}
\label{sec:intro}
Photometry for Orbits, Eclipses, and Transits, or POET, is a Python
code for modeling light curves of transiting exoplanets. It is
end-to-end, taking in raw images, performing centering and photometry,
and modeling the transiting light curve, including systematics
effects.  It has models for the Spitzer ramp and intrapixel
systematics, and several transiting planet models with many different
parameters. Model parameter uncertainties are accurately estimated
with a Markov-chain Monte Carlo routine.

This document is intended to provide a user with the information necessary
to run POET and interpret the results.

\section{Code Structure}
\label{sec:structure}
POET is modular, separated into sections labeled p1 -- p10 and Zen
Eliminates Noise (ZEN), an implementation of Pixel-Level Decorrelation
(PLD, see Deming et al., 2015).  The modules functions are reading the
data (p1), bad pixel masking (p2), centroiding (p3), photometry (p4),
checks and plots (p5), light-curve modeling (p6 and ZEN),
post-modeling analysis (p7), making tables (p8), making figures (p9),
and writing light curves to files (p10).

Note that POET can apply both BiLinearly Interpolated Subpixel
Senitivity maps (BLISS, see Stevenson et al., 2012) and other pixel
maps or PLD. Because BLISS is multiplicative and PLD is additive, they
are handled by separate modeules of the code. All users should run p1
-- p5, and then decide if they want to use a PLD or BLISS model and
run ZEN or p6 -- p10 accordingly. We recommend using both to check for
consistency in results.

Every notable variable created within POET is saved in the event
object (event). Frame parameters, such as centering position and
photometry, are contained in a frame parameters object within the
event object (event.fp), which is populated by p1 through p5.  The
results of a model fit are saved in a fit object within the event
object (event.fit), created by p6 and ZEN. The event object can be
loaded into a Python session at any point in the analysis process
using either the manageevent.py module or the run.py module, depending
on how much of the pipeline has been run.

\subsection{p1 - Reading the Data}
\label{sec:p1}
POET begins with p1. This module initializes the event object with
the input values and reads the data files, the FITS images, into
the event object, including header information. This is all handled
by the initialization of the event object.

p1 produces an image of the mean frame, which is the brightness
averaged at each position for the entire observation. This figure,
labelled *-fig101.png, shows the estimated location of the target with
a red '+'. It is important that this mark lie exactly on top of the
brightest pixel in the star. If not, the user must update the source
estimate using the centering configuration in the manner described in
Section \ref{sec:procedure}.

\subsection{p2 - Bad Pixel Masking}
POET uses the masks for permanently bad pixels that come with the data,
but there are still transient bad pixels to find, such as cosmic ray hits.
POET performs two iterations of 4$\sigma$ rejection by default, although
this can be changed in the POET config files (pcf).

POET splits the data up into chunks, usually of 64 frames although this
can be edited in the pcf. It then takes the median of the pixel values at
each (x,y) location on the array, then masks pixels outside 4 standard
deviations of the median for that location. This median and standard
deviation take into account the current mask, so iterations beyond the
first can find additional bad pixels.

The user can also mark specific pixels to be masked out in the pcf.

\subsection{p3 - Centering}
\label{sec:p3}
This module finds the center of the target on the detector array at
subpixel precision. This level of precision is necessary because of
intrapixel effects introduced by subpixel sensitivity variations. The
magnitude of these variations is greater than the strength of an
eclipse signal, so accurately removing them is of utmost importance.

There are several centering methods available:

\begin{itemize}
\item Fit Gaussian Centering (fgc) -- this method fits a 2D Gaussian
  to the pixel values within a given circular radius around the peak pixel.

\item Center-of-Light (col) -- this method weighs each pixel location
  by its measurement, and determines the peak location as the
  weighted average, akin to a center-of-gravity calculation. It considers
  pixels in a square around the peak pixel with radius specified in
  the pcf.

\item Circular Center-of-Light (ccl) -- same as col, but uses a circle
  instead of a square aperture.

\item Least-Asymmetry Gaussian (lag) -- transforms the pixel values
  into an array of symmetry values. Symmetry values are calculated based
  on the pixels surrounding the position considered, with a square of
  radius specified in the pcf. This is done for every pixel in a square
  of (a different) radius around the peak value, also specified in the pcf.
  It then fits an inverted Gaussian to the symmetry values to determine
  the position of least asymmetry. See Lust et al. (2014).

\item Least-Asymmetry Center-of-Light (lac) -- same as lag, but uses
  the center-of-light algorithm rather than a Gaussian fit to the symmetry
  values.

\end{itemize}

Fgc, col, and lag are the primary centering methods usually considered.

\subsection{p4 - Photometry}
\label{sec:p4}
The goal of photometry is to measure the brightness of the target.  In
the most basic sense, photometry is the sum of pixel values within an
aperture. POET performs interpolated, flux-conserving photometry.  The
pixels are divided into subpixels (default 5x5 subpixels per pixel),
interpolating the pixel values while maintaining the total flux. It
then sums the subpixel values within an aperture of specified
shape and size. There are currently four photometry methods available:

\begin{itemize}
\item Constant Aperture -- this method uses the same aperture radius
  for every frame. Radii usually range from 1.5 to 5.0 pixels, with the
  best size in the 2.0 to 3.0 pixel range.

\item Variable Aperture -- this method varies the aperture radius
  for every frame. The aperture radius is calculated from the
  noise-pixels value (see Lewis et al. 2013 and the Spitzer handbook).
  The radii are the square root of the noise pixel value for each frame
  plus a constant offset.

\item Elliptical Aperture -- like variable apertures, this method
  uses a different aperture size for each frame. However, the aperture
  is elliptical, using the width of the PSF in X and Y, calculated by
  fitting a Gaussian in p4. Consequently, this method only works
  with Gaussian centering.

\item Optimal -- this method fits a PSF to each frame and then
  determines the 'optimal' photometry one should be able to achieve.
  Typically this is reserved for IRS peak-up camera and MIPS photometry.

\end{itemize}

The user should run a large variety of photometry apertures and then
determine the best during the modeling section (see Section \ref{sec:p6}).


\subsection{p5 - Checks}
\label{sec:p5}
This module creates diagnostic plots and performs a few calculations
in preparation for modeling (light-time correction, leap second correction,
position RMS, etc.). 

The figures are as follows:

\begin{itemize}
\item Fig501 -- Binned flux vs.\ Time. Simply the photometry plotted against
  time, binned to a reasonable number of points. This plot gives the user
  an idea of the shape of the ramp in the photometry and the level of the
  intrapixel effect (see Section \ref{sec:p6}).

\item Fig502 -- Binned flux vs.\ Pixel Position. This plot shows how the
  measured brightness of the target changes with pixel position, which
  shows the severity of the intrapixel effect for this data set.

\item Fig503 -- Pixel Position vs.\ Time. This plot shows how the
  target moves on the detector as time progresses, in both x and y. It
  can be helpful in identifying rapid motion of the star, such as when
  the telescope is adjusting its reaction wheels. Often, images
  captured during such an event must be discarded.

\item Fig504 -- Flux vs.\ Radial Distance. The brightness of the target
  as it varies with distance from the center of the pixel. Similar to
  Fig502, except the two plots have been combined into one.

\item Fig505 -- Background Level vs.\ Time. The calculated background level
  which is subtracted from each respective frame. This should be relatively
  constant. Activity in the background level may indicate bad frames.

\item Fig507 -- Noise Pixels vs.\ Time. The noise pixel parameter, calculated
  according to Lewis et al. (2013). If using variable apertures, this plot
  gives the user an idea of how the aperture radius is changing with time.
  Any sudden changes in the noise pixel parameter have been shown to
  impersonate transit features, so the user should check their end results
  against this plot to identify possible false positives.

\item Fig508 -- Noise Pixel Variance vs.\ Time. This plots the variance
  of the noise pixel parameter over each subarray set number (64-frame
  chunks). Spikes in this plot coincide with PSF activity. If they occur
  during an astrophysical event, the event may actually be due to
  telescope vibration. Consider clipping frames which correspond
  to high noise pixel variance, or using elliptical photometry.

\item Fig509 -- Gaussian Elliptical Area vs.\ Time. This plot is
  only produced if Gaussian centering or a variant on that method
  are used. Elliptical area is the area of an ellipse with axes
  equal to three times the Gaussian width in x and y, measured from
  p3. Like Fig508 and Fig509, spikes indicate PSF activity.

\item Fig510 -- Gaussian Elliptical Area Variance vs.\ Time. Interpreted
  the same as Fig508, but spikes can be more evident in this plot.

\end{itemize}


\subsection{p6 - Modeling}
\label{sec:p6}
The brunt of the work is done here, or in ZEN. p6 combines
transiting planet models, ramp models, and pixel maps to create a
model to compare to the photometry. It begins with a $\chi^2$
minimization of the model, and then runs an MCMC routine to accurately
determine model parameter uncertainties. The models of most importance
are:

\begin{itemize}
\item mandelecl -- An eclipse model based on Mandel \& Agol (2002).
  Parameters are eclipse midpoint, width, depth, ingress time, egress
  time, and system flux.

\item mandeltr -- A transit model based on Mandel \& Agol (2002).
  Essentially identical to mandelecl, except depth is the cross-sectional
  area ratio of the planet to the star, rather than the relative brightness
  of the planet to the star.

\item linramp -- A linear ramp model. The user should consider including
  this model if the photometry appears to follow a linear baseline.

\item quadramp -- A quadratic ramp model. Similarly, this should be
  considered if the photometry follows some curvature.

\item bilinint -- This is the BiLinearly Interpolated Subpixel
  Sensitivity map (BLISS), which removes the subpixel sensitivity
  variation from the photometry. It has no parameters, as the
  sensitivity variations are all pre-calculated based on the
  photometric variations. All p6 science runs should include this
  model unless there is good reason to believe there is no intrapixel
  sensitivity variation.
\end{itemize}

There are dozens of models for various scenarios. For a full list of them,
see models.txt.

p6 produces many plots. The following is a list of them, and how to interpret
them:

\begin{itemize}
\item Fig6*01 -- Binned data with the model and model without the
  transit/eclipse over-plotted. The model should follow the data. If not,
  something has gone quite wrong. See other plots to diagnose.

\item Fig6*02 -- Binned data with the model over-plotted, but non-eclipse
  and non-transit models have been divided out. This should make the
  feature clear. If you still see a trend in the baseline, consider changing
  the ramp model. If there is still variation due to the intrapixel effect,
  consider changing the bin size of the BLISS map.

\item Fig6*03 --Trace plots of the various model parameters during the
  MCMC, which show the value of each parameter as they change at each
  step of the MCMC. Each parameter should be clustered around a
  central average. With enough iterations, they should look like a
  fuzzy blob. If there appears to be a pattern in the trace plots at
  the early iterations, consider increasing the number of burn-in
  iterations. If there is a pattern throughout, you may have a
  multimodal or some other other complex posterior distribution;
  consult Fig6*05.

\item Fig6*04 -- Autocorrelation plots of model parameters. These should
  start at 1 (as a data set is completely correlated with itself) but drop
  to 0 quickly. If not, the effective step size of that model parameter
  is relatively large compared to the total number of iterations. The user
  should then consider increasing the total number of iterations in the
  MCMC.

\item Fig6*05 -- Correlations between model parameters. Ideally, these
  should appear roughly Gaussian (peaks in the center, circularly
  symmetrical).  There may be strong correlations between
  parameters. If a correlation plot has no cohesive shape, this may
  indicate very little constraint on a model parameter, in which case
  the user should consider putting a prior on that parameter, either a
  Gaussian or fixed, from the literature or a reasoanble calculation
  (e.g., calculating expected eclipse duration from orbital
  parameters).

\item Fig6*06 -- Histograms of model parameters. These should also be
  roughly Gaussian. If they appear flat, then that parameter is not well
  constrained and the user may want to place a prior on it. If a histogram
  is shoved up against the edge of a plot, the parameter is trying to
  go outside the set boundaries. Consider adjusting the boundaries on
  that parameter within reason (e.g. eclipse depth cannot go below 0).

\item Fig6*07 -- BLISS map sensitivity. This plot shows the subpixel
  sensitivity model used for this data set. The BLISS map should match
  the binned flux.

\item Fig6*08 -- BLISS map in 2D. There should be relatively smooth
  changes in sensitivity with position.

\item Fig6*09 -- Pointing histogram. This should look roughly
  Gaussian. If there are outliers, the user may want to consider
  clipping those points, as the BLISS model is ineffective in areas
  without a large number of samples.

\item Fig6*11 -- Correlated noise in the model residuals. This plot
  shows the fit RMS as a function of bin size. Ideally the black line
  should follow the red line. If the black line is significantly higher
  than the red line, there is evidence for correlated noise at that binning
  level. The blue dotted line indicates a binning level equal to the
  ingress/egress duration, and the green dashed line indicates a binning
  level equal to the eclipse/transit duration. Correlated noise
  could indicated an unoptimized BLISS map, although sometimes it cannot
  be completely removed.

\end{itemize}

p6 is an iterative process of determining best parameters' initial values,
and boundaries, which models to use, and the best centering/photometry
combination. For a discussion of this, see Section \ref{sec:procedure}.

\subsection{p7 - Analysis}
\label{sec:p7}
This module performs some additional calculations based on the output of p6.
Calculations include brightness temperature (for eclipses), ephemerides
in Julian days, orbital parameter estimates, and correlations. In most cases,
this can simply be run after p6 without any additional work for the user.

\subsection{p8 - Tables}
\label{sec:p8}
This module makes tables, which are formatted for LaTeX.

\subsection{p9 - Figures}
\label{sec:p9}
This module creates figures for papers. They are all previously made
plots, but a bit fancier.

\subsection{p10 - Light Curve Tables}
\label{sec:p10}
This module produces IRSA tables and FITS files containing the light
curves that were modeled. These are meant to be submitted in support
of publications.

\subsection{p11 - Advanced Figures}
\label{sec:p11}
This module creates some even fancier plots. It is written to be edited
according to the user's needs.

\subsection{p12 - Advanced Tables}
\label{sec:p12}
This module creates some more LaTeX tables for papers. Note that it is
run separately from poet.py.

\subsection{ZEN - Zen Eliminates Noise}
\label{sec:zen}
This module, originally a standalone code, is an implementation of
Drake Deming's Pixel-Level Decorrelation method (PLD, see his 2015
paper).  It runs an MCMC with the PLD function, using MC$^3$
(github.com/pcubillos/mccubed, Cubillos et al. 2017). ZEN replaces p6
through p10, as it does modeling, calculates temperature, makes
figures, and saves the FITS/IRSA files.

ZEN has three main parts: centering, photometry, and bin size
optimization; MCMC; and post light-curve-modeling analysis. First, ZEN
loops over all given centering methods, photometry aperture sizes, and
specified bin sizes to determine the optimal combination. Then, it
runs MCMC on the best combination to determine parameter
uncertainties. Finally, ZEN calculates planetary temperatures,
eclipse/transit times in BJD, and creates light-curve
files. Effectively, this combines the functionality of p6, p7, and
p10, but using a PLD model rather than a sensitivity map.

ZEN produces the following figures:

\begin{enumerate}
  \item *-normlc -- Normalized light curve with model
    overplotted. Systematics have been subtracted out.

  \item output\_zen\_model -- Non-normalized light curve with model
    overplotted. Systematics not subtracted out.

  \item output\_zen\_posterior* -- Posterior histograms for fitted
    parameters. See Fig 6*06 of p6.

  \item output\_zen\_pairwise* -- Pairwise posterior histograms for
    fitted parameters.  See Fig 6*05 of p6.

  \item output\_zen\_trace* -- Trace plots of fitted parameters. See
    Fig 6*03 of p6.

  \item *-logbsig -- A plot of log(binned-$\sigma$ $\chi^2$) values
    for each centering, photometry, and bin size combination
    considered. The code will tell you the combination which gives the
    best results, but this plot can help identify other potential
    combinations.

  \item *-chislope -- A plot of the slope of the binned-$\sigma$
    relation for each centering, photometry, and bin size combination
    considered. Like the logbsig plot, this helps identify other
    potential combinations.

  \item *-pixels -- A plot of the pixels used in the PLD model, with
    numbers corresponding to the same parameter numbers in the other
    plots. The marked pixels should be clustered near the target star.

\end{enumerate}

See Section \ref{sec:procedure} for a discussion of how to run the
ZEN module.

\section{Configuration Files}
POET uses almost exclusively ConfigParser
(https://docs.python.org/3/library/configparser.html) format
configuration files. There are two important configuration files:
eventname.pcf and eventname-zen.cfg.

\subsection{eventname.pcf}

This file contains all configuration options for p1 -- p7, as well as
the denoising options. It is split into several sections, separated by
a header between brackers ([]). Typically, a single section contains
the options for a single modules with some exceptions: p2 options are
in the [event] section, only includes the chunk size and
sigma-rejection settings; p5 has no options; and p7 options are
included in the [params] section, which only includes a number of
temperature calculations.

Most configuration options take a single value. However, some options
can be lists, or lists of lists, in which case the corresponding
module will be executed for each possible combination of settings. For
example, if the user specifies three photometry methods, two inner sky
annulus radii, and two outer sky annulus radii, p4 will be executed 12
times in parallel, and each of these processes may take many CPU cores
(depending on the ``ncores'' option) and a significant amount of
RAM. Users must be careful about how many resources their runs demand.
Some notable settings that have this behavior: centering method,
photometry method, photometry aperture scale, photometry aperture
offset, and sky annulus radii.

Occasionally, you may want to run several different configurations in
parallel, but not every possible configuration. In that case, the user
can create new sections titled [centering2], [photometry3], etc. Within
these sections, all combinations will be run, but the code will not run
every possible unique setting combination. For example, if the following
is specified in the [photometry] section:

\begin{verbatim}
[photometry]
phottype aper var
...
photap 1.0 2.75
offset 0
...
\end{verbatim}

\noindent
will run aperture photometry and variable photometry each for every
combination of ``photap'' and ``offset'', for 4 total executions. However,
if the user instead creates two sections like the following:

\begin{verbatim}
[photometry]
phottype aper
...
photap 1.0
offset 0
...
[photometry2]
phottype va
...
photap 2.75
offset 0
...
\end{verbatim}
\noindent
then the code will still run both aperture photometry and variable
photometry, but with only one combination of ``photap'' and ``offset''
each, for a total of two executions. Note that all section names must
be unique, and must be named to match the section format they follow
(e.g., a photometry section must be named [photometry\#]).

The specific options are discussed in further detail in Section
\ref{sec:procedure}.

\subsection{eventname-zen.cfg}

This file contains all options for ZEN executions. There are only two
sections: [EVENT] and [MCMC]. The [EVENT] section contains
event-specific options, and [MCMC] provides settings for MC$^3$. The
specific options are discussed in further detail in Section
\ref{sec:procedure}.

\section{Procedure}
\label{sec:procedure}
This section outlines the general procedures to follow in a single,
complete POET run.

Make a directory for your planet analysis in /home/esp01/events. This is
where all your runs take place. Replace PLANETNAME and USERNAME with the
proper information.

\begin{verbatim}
mkdir /home/esp01/events/PLANETNAME-USERNAME
cd /home/esp01/events/PLANETNAME-USERNAME
\end{verbatim}

Next, pull POET from the local git repository. Replace DATE with the
current date so that you know which version of POET you cloned, and so
that directories will list in chronological order.

\begin{verbatim}
git clone --recursive /home/esp01/git/poet.git DATE
cd DATE
\end{verbatim}

Before doing anything else, we need to compile some of the C
code. This includes the centering algorithms, the least asymmetry
code, the MCcubed module, as well as various models written in C. Run
the following to build the code, and compile the user manual:

\begin{verbatim}
make
\end{verbatim}

Make subdirectories for each data set you plan to analyze. Naming conventions
for event codes are formatted as follows:

\begin{itemize}
\item Two letter survey code. Options are:
  \begin{itemize}
  \item ha -- HAT-P survey
  \item hd -- Henry Draper catalogue (e.g. HD209458)
  \item tr -- TrES survey
  \item wa -- WASP survey
  \end{itemize}

  \item Three number star designation. Always use exactly three numbers (e.g.
    for WASP-12, use 012, or for HD209458 use 209).

  \item The letter of planet (usually b, unless your system has many planets).

  \item A letter designated the type of event. ``p'' for primary transit,
    ``s'' for secondary eclipse, or ``o'' for orbit.

  \item A number designating the Spitzer IRAC channel. Use 1 through 4 for
    the primary channels, 5 for the blue peak-up array, and 6 for the MIPS
    24um array.

  \item A number designating the visit number.
\end{itemize}
    
For example, if you have one eclipse observation with Spitzer channel
2 and two eclipse observations with Spitzer channel 1 of exoplanet
WASP-12b, then you would do the following:

\begin{verbatim}
mkdir wa012bs11
mkdir wa012bs12
mkdir wa012bs21
\end{verbatim}

You will need to get config files for your runs. Copy them from the example
run/ directory into your new directories. For example:

\begin{verbatim}
cd wa012bs12
cp ../run/*.pcf ../run/*.cfg ../run/*.txt .
\end{verbatim}

You will need to edit all these files before running. Next, edit the
file eventname.pcf. First, it MUST be renamed according to your event
code. For the second observation in channel 1 of our example
observations, the command would be

\begin{verbatim}
mv eventname.pcf wa012bs12.pcf
\end{verbatim}

\noindent
{\underline{\bf NOTE}}: This file will alawys be referred to as
eventname.pcf, even though it will be renamed as above

Now, open the renamed file with your favorite text editor, which is
Emacs.  You will notice sections with headers designated by
brackets. The first of these is the ``event'' section. In this section,
the following parameters need to be changed according to your data.

\begin{itemize}
\item tepfile -- This is a Transiting Exoplanet file. They can be
  found in /home/esp01/obsprep/TEP. If the file you need does not
  exist, you will need to make one. TEP files must follow very
  specific criteria, which you can read about in
  /home/esp01/doc/README\_TEPfiles.

\item datadir -- This is the name of the directory which holds the
  data for this analysis. The location will be relative to topdir, a
  user variable that is usually left unchanged from its default,
  /home/esp01.  For WASP-12b channel 1 visit 2, the datadir would be
  data/2010-Spitzer-too/wa012bs12.

\item sscver -- This is the subdirectory in the data directory which
  corresponds to the Spitzer pipeline version used for processing the
  raw images. For our example, this is S18.14.0. You should make sure
  your data was processed with the latest version of the pipeline
  (19.2.0 for \textit{Spitzer} IRAC as of this writing).

\item aorname -- This is the number corresponding to your datasets
  AOR. Within the sscver directory, there will be multiple
  mostly-numeric directories.  You need to determine which AOR
  contains the data. To do so, check the bcd directory within each AOR
  directory. The directory with the data will have far more files than
  the others. Set the aorname to the number within the name of the AOR
  directory. For the example, this would be 39200768. You can also
  check the sizes of the AOR directories (e.g., with the 'du' utility)
  or looks up your observation on the Spitzer Heritage Archive to
  determine which ones contain the science observation (typically much
  longer than calibration observations).

\item aortype -- This tells POET the type of images in the AOR. Set
  this to 0, since we are only including the data AOR.

\item pmaskfile -- This is a fits file which contains the locations of
  permanently bad pixels. It can be found in the data directory, and
  should have a name like *pmask.fits. For our example, it's
  sep07\_ch1\_pmask.fits.

\item kuruczfile -- Kurucz files contain stellar spectra. You'll need
  one which best represents the star your planet orbits. First, check
  /home/esp01/ancil/kurucz/ for a file with your planet's name. If one
  does not exist, you will need to fetch them from Kurucz's
  website. See the file /home/esp01/doc/info\_kurucz for more
  information. For our example, the file is wasp12b-fp02k2odfnew.pck.

\item psffile -- This can be left as default unless you are not using
  IRAC data.

\item runp2 and runp3 -- Setting these to True will make POET
  automatically run p2 and p3 after it finishes p1. If you do so, make
  sure that you have configured the input files for those modules
  before running p1.

\end{itemize}

Once you have configured the ``event'' section of the file, you may
choose to run EDGAR (Easy Distrubuted Global Aggregator of Runs),
which allows for easier execution of POET but less customization. If
you would like to use EDGAR, read the sections below about filling out
p10.pcf and initvals.txt and skip to the section
\ref{sec:POE}. Otherwise, continue reading for further instructions on
setting up your analysis.

%% The next three sections (denoise, centering, photometry) are read
%% slightly differently. If a section has a parameter with multiple space
%% separated values, POET will be run once for each combination of
%% parameter values. For instance 

%% Lastly, the output directories are named based on particular parameters:\\
%% \begin{tabular}{|c|c|c|}\hline
%%   POET code & description & output name \\\hline
%%   pd & denoise & $<$wavelet$>$\_$<$threshold$>$\_$<$numlvls$>$\\\hline
%%   p3 & centering & $<$method$>$\\\hline
%%   p4 & photometry & $<$phottype$><$photap$><$offset$>$*\\\hline
%% \end{tabular}\\
%% * The photometry output directory name will be just ``optimal'' or ``psffit'' if one of those is selected for phottype.\\\\
%% If you want to run POET with two configurations of parameters which
%% result in the same directory name, you will need to use pcfnaming. You
%% can always use the pcfname parameter, but in this case you
%% must. Setting `pcfname` to something other than ``None'' will append
%% an underscore and the value of pcfname to the name of the output
%% directory.

Next, you need to fill in the centering section. The following
describes each option and potential choices:

\begin{itemize}
\item pcfname -- This is a string that will be appended to the output
  directory name. Output directories are named according to the ``method''
  option. If you want to, for example, run ``fgc'' centering with two
  different sets of options, you will need to set pcfname to something
  other than None or the output of one run will overwrite the other.
  
\item method -- This is the centering method(s) to be used. Options are
  discussed in Section \ref{sec:p3}. Note that you can list multiple methods
  here, separated by spaces, and POET will run them all at once.

\item ccores -- POET will use this number of cores for each centering
  method.  Be careful that you do not use too many cores. Always check
  CPU availability (e.g., with htop) before starting a cpu-intensive
  run.

\item noctr -- A boolean. If True, the code will only center the mean
  image and assume that center position for all frames. This should
  be False in almost all scenarios.

\item ctrim -- This is the radius of the trimmed box around the
  star. We use 8 pixels by defaults, but some testing has shown that
  changing this number can drastically effect the centering, and thus
  the photometry and the model fitting. If you find peculiar centering
  results, consider adjusting this parameter.

\item weights -- A boolean to use uncertainties as weights for fits.
  Typically this is left False.

\item fitbg -- A flag with options of 0, 1, or 2 for background
  fitting when doing Gaussian centering. Setting 0 will not fit a
  background, and just use a median. Setting 1 will fit a constant
  background along with the Gaussian fit to the star. Setting 2 will
  fit a plane background along with the Gaussian fit. Typically a
  constant background is sufficient.

\item runp4 -- Setting this to True will cause p4 to run immediately
  after p3.  Only do so if you have configured the photometry section,
  and keep in mind that p4 will be run on every centering method in
  parallel, which can a significant number of CPUs and amount of RAM.

\item cradius -- This is the size of the box around the star
  considered when doing least-asymmetry centering. The code will try
  to use the user input value, but sometimes must adjust to make the
  calculation possible.

\item csize -- THis is the size of the box around each pixel when
  calculating an asymmetry value for least-asymmetry centering. The
  code will try to use the user input value, but sometimes must adjust
  to make the calculation possible.

\item psfctrim -- Same as ctrim but for finding the center of the PSF.

\item psfcrad -- Same as cradius but for finding the center of the PSF.

\item psfcsize -- Same as csize but for tinding the center of the PSF.

\item npskyin, npskyout -- Radii of the sky annulus for background
  subtraction prior to noise pixel calculation. Should be the same as
  you intend to use for sky calculation in photometry.

\item ymask, xmask -- Specific pixels to mask out prior to centering.
  This must be comma separated without spaces.

\item newpsf -- Optional PSF file to use instead of the one given in the
  [event] section.

\item expand -- The expansion factor of the given PSF. Should be 5
  for the \textit{Spitzer} PSF.

\item nopsfctr -- Boolean to not do PSF centering, and assume the center of
  the PSF is the center of the frame. Typically leave this False.

\item srcesty -- Adjust this to match the y position of the star on
  the detector, if the original estimate is off. Check *-fig101.png
  (produced by p1). If the red '+' is on the center of the star, you
  may leave this parameter alone. Otherwise, adjust as necessary. An
  incorrect source estimate position can adversely affect your
  centering and photometry results.

\item srcestx -- Same as srcestx, except in the x dimension.
\end{itemize}

Now edit the photometry section.

\begin{itemize}
\item pcfname -- This is a string that will be appended to the output
  directory name. Output directories are named according to the
  ``phottype'', ``photap'', and ``offset'' parameters combined into a
  single string. For example, if ``phottype'' is aper, ``photap'' is
  2.5, and ``offset'' is -1.0, then the output directory will be
  ap250-100. Like with centering, if you run two photometry configurations
  that result in the same output directory names, the output will be
  overwritten. Setting pcfname will change the output directory name to
  prevent output overwrite.
  
\item phottype -- This is the photometry method. There are three main
  choices: aper, for constant circular aperture photometry; var, for
  variable circular aperture photometry (see Lewis et al., 2013); and
  ell, for variable elliptical aperture photometry (see Challener et
  al., 2019, in prep). There is also an ``optimal'' option, which
  attempts to fit a PSF to the image and determine the optimal
  total photometry one could achieve. This option is typically
  reserved for IRS peak-up camera and MIPS observations.

\item ncores -- This the number of cores used FOR EACH APERTURE. This
  can easily get to be a large number of cores, so be very careful. At
  the moment, photometry is inefficient with memory usage, duplicating
  the images across every core used, so pay attention to your RAM
  while running photometry. Also, check htop before starting a large
  run.

\item denphot -- Boolean. If True, use denoised data (produced by the
  denoising module, pd). Typcally left False.

\item photap and offset -- These are the photometry aperture radii
  scalings and offsets that POET will run. In general, when
  calculating aperture size for each frame $i$, POET uses the formula
  ${\rm radius}_i = {\rm photap} \times {\rm base}_i + {\rm offset}$,
  where ${\rm base}_i$ is set by ``phottype''. If ``phottype'' is
  aper, ${\rm base}_i$ is 1 for all frames, so all frames use the same
  aperture size. If ``phottype'' is var, ${\rm base}_i$ is the square
  root of the noise pixels of each frame, a calculation described in
  Lewis et al., 2019. If ``phottype'' is ell, ${\rm radius}_i$ is
  calculated once with ${\rm base}_i$ as the Gaussian $x$ width of the
  target, and once with the Gaussian $y$ width of the target. These
  two ${\rm base}_i$ are used to calculate two ${\rm radius}_i$ which
  define the respective $x$ and $y$ ellipse widths for the elliptical
  aperture. Note that POET will run every combination of photap,
  offset, and phottype given, which can use a significant number of
  CPUs and RAM.

\item skyin, skyout -- The radii of the annulus, in pixels, that will
  be used to determine the sky level. Note that POET will run all
  combinations given. Typically, 7 and 15 work well, but adjustment
  may be necessary if there are other nearby stars.

\item skyfrac -- Frame will discarded if less than this fraction of
  sky pixels are marked as good pixels.

\item skymed -- Boolean for sky estimation method. If True, the code will
  take a median of all pixels in the sky annulus. If False, it will use
  a mean.

\item verbose -- Verbosity boolean.

\item apscale -- Images will be interpolated by this factor before
  photometry is done. 5 works well for \textit{Spitzer}.

\item psfexpand -- The expansion factor of the given PSF. Should be 5
  for the \textit{Spitzer} PSF.

\item runp5 -- Setting this to True will immediately run p5 after p4
  completes. This is usually safe, as p5 has no configuration.

\item otrim -- Radius of window around the target to consider for optimal
  photometry.
\end{itemize}


Finally, edit poet.pcf, which is in your top level
directory. Currently, all you need to edit is the run directory and
the event code in the POET section of poet.pcf, which should both be
set to one of the name of the directory you made earlier
(e.g. wa012bs12 for our example) which contains the other .pcf file
you just filled in.
\\In poet.pcf:\\
\begin{verbatim}
[POET]
rundir    wa012bs12
eventname wa012bs12
...
\end{verbatim}

Now that you've set everything up, you can run p1 with the following
command from the run/ directory:

\begin{verbatim}
poet.py p1
\end{verbatim}

If you set runp2, runp3, runp4, and runp5 to True, then this will run
all the way through p5. You can run each step individually as well,
with the following commands, which is recommended for new users:

\begin{verbatim}
poet.py p2
poet.py p3
poet.py p4 CENTDIR
poet.py p5 CENTDIR/PHOTDIR
\end{verbatim}

Make sure to replace CENTDIR and PHOTDIR with the proper names.  This
should not include the ``run'' directory (wa012bs12 in the example) so
CENTDIR should be something like ``fgc'' and PHOTDIR should be
something like ``ap250+000''.

There are two ways to continue with the analysis, and you should
do both:  ZEN, and p6 through p10. First the remaining p* modules
will be described, and then ZEN. Keep in mind that ZEN
is a replacement for p6 through p10.

Once you have run all the centering and photometry methods that you
would like, you can move on to p6, where the real work begins.

In my experience, the following is the best way to go about modeling
light curves. Others may find that something else works best for them,
and it may vary between datasets. Here is the procedure:

To begin, you will need to edit the params section of the renamed
eventname.pcf (wa012bs12.pcf in the example):

\begin{itemize}
\item planetname -- This string will be used for plot titles and such.
  Edit this to be your planet name as you want it to appear on plots.

\item modelfile -- This should be the name of the initivals file, which
  should be eventname-initvals-bliss.txt. In the example, this
  is wa012bs12-initvals-bliss.txt.

\item printout -- The name of the file which will contain the results. The
  default is fine.

\item timeunit -- Units of time. Generally use 'orbits'.

\item tuoffset -- Offset in the time units. Useful if using Julian days
  (e.g. 2450000).

\item model -- This is where you choose which models will be included
  in the total model. See /home/esp01/doc/models.txt for a complete
  list.  The most common are listed in Section \ref{sec:p6}. Note that
  this is interpreted as a list of lists where each line is a
  list. Each line after model (that should be interpreted as a list of
  models) must be indented by at least one space. If you use an
  intrapixel map, it must the last model in the line.

\item mcmc -- A flag that toggles the use of MCMC. Setting this to
  False can significantly decrease runtime, but you will not get
  uncertainties on model parameters and many of the p6 figures will
  be useless. Note that in the code, this toggle amounts to filling
  the MCMC posterior array with zeroes, and setting the best-fitting
  parameters to the results of the least-squares fit. Toggling off
  both MCMC and least-squares is not recommended.

\item numit -- Two-element integer array, with the first being the number
  of burn-in iterations and the second being the number of total iterations.
  Typical numbers for early runs are 1e3 burn-in, 1e4 full. A final run
  should use more, such as 1e4 burn-in, 1e5 total. At the very least, you
  want to make sure that your run converges.

\item nchains -- Number of parallel MCMCs. 6 is generally sufficient
  when using the Snooker random walk algorithm.

\item rwalk -- MCMC random walk algorithm. Options are 'mrw', 'demc',
  and 'snooker'. Generally use 'snooker'. DEMC and Snooker adapt the
  step size as the chains progress, so that the user does not have to.
  Snooker converges the fastest, and requires less chains and thus,
  less processing power.

\item nbins -- Number of points on the resulting plots. 60 is clear,
  although sometimes more is preferred to see shorter-timescale
  features.

\item boundecl -- Leave this as True to force the eclipse to stay
  within the data range. This should not have to be changed unless you
  suspect your observation missed the eclipse.

\item chi2flag -- Boolean to adjust error bars such that reduced
  chi-squared is forced to 1. Spitzer overestimates errors, so we use
  this to scale them down to reasonable values. Leave this as True,
  except when comparing ramp models, as the rescaling forces a good
  fit when there may be none.  For example, if there is an obvious
  ramp effect in the light curve but your model does not include a
  ramp model.

\item numcalc -- Number of Monte Carlo temperature calculations in
  p7. This can take a long time with a large number of calculations,
  but you only have to run p7 once. The Monte Carlo pulls values from
  the MCMC posterior, so if numcalc $>$ numit, the code reduces
  numcalc to make the calculation possible.

\item thinning -- This is the step size for histogram thinning. If you
  have a lot of iterations, you may want to raid this number (and similar
  for less iterations).

\item allplots -- Boolean for additional plots to be made. Making this
  False can decrease run time.

\item leastsq -- Perform least-squares fit before the MCMC. Leave this as True
  unless you are having issues with the least-squares fit.

\item normflux -- Normalize flux at each position if there are more than 1
  positions. In most cases, this parameter will have no effect.

\item noisysdnr -- SDNR of noisy data set. Leave this as None, as we no longer
  use denoised data.

\item preclip -- Number of frames to remove from the beginning of the
  data set. If you notice a strong trend in the beginning of the data
  set, or some of the data seem strange, you may consider clipping
  them.

\item postclip -- Same as preclip, except the end of the data set.

\item interclip -- Same as preclip, except you can specify multiple
  ranges of data points to remove (same format as models).

\item xstep and ystep -- Bin size of the BLISS map in x and y
  respectively. Leaving these as 0's will set them to the RMS of
  the point-to-point variation in the position of the target in x
  and y, respectively. This is generally sufficient.

\item minnumpts -- Minimum number of points per bin in the BLISS
  map. It should be at least 4 for your final run (see Stevenson et
  al. 2012).

\item isfixipmap -- Boolean to fix BLISS map to best-fit values after
  burn-in. Generally leave this False.

\item ipclip -- List of ranges of points to remove from intrapixel
  mapping. Generally unnecessary.

\item issmoothing -- Boolean to turn smoothing on or off. This
  convolves a Gaussian kernel with the image. I recommend leaving as
  False.

\item nx and ny -- Half-width of the Gaussian kernel in units of pixels
  in x and y, respectively. Unused if issmoothing is False.

\item sx and xy -- Standard deviation of the Gaussian kernel in units
  of pixels in x and y, respectively. Unused if issmoothing is False.

\item xdiv and ydiv -- Locations in subpixels to divide pixels into
  quadrants.  Only used with quadip4, a model we no longer use. Leave
  as 0.0.

\end{itemize}

Next, you'll want to edit the initvals file. This file contains a list
of all the available models along with their parameters, parameter
minima, parameter maxima, and step-sizes. Rename the initvals with
your planet code like so:

\begin{verbatim}
cd run
mv initvals.txt wa012bs12-initvals-bliss.txt
cd ..
\end{verbatim}

The code will run without renaming this file, but it is much easier
to keep track of your runs if this filename is changed this way, and
joint fits require the files to be renamed.

Navigate to the eclipse and ramp models that you selected in your
params file. For the eclipse model, you'll want to make sure that
there are reasonable bounds on your parameters. Generally, you'll want
to begin with wide boundaries and narrow them as necessary. For
example, if using mandelecl you might have a few lines like this:

\begin{verbatim}
mandelecl
midpt     	width     	depth     	t12       	t34       	flux
5.0108e-01	4.6942e-02	2.9199e-04	3.3613e-03	3.3613e-03	5.2850e+05
4.7000e-01	1.0000e-04	0.0000e+00	1.0000e-03	1.0000e-03	0.0000e+00
5.3000e-01	1.0000e-01	5.0000e-02	8.0000e-02	8.0000e-02	1.0000e+07
1.0000e-04	1.0000e-04	1.0000e-05	1.0000e-05	-4	5.0000e+00
\end{verbatim}

Be sure that the top row of any model, the initial values, lies
between the numbers in the second and third row (the minimum and
maximum for that parameter, respectively). Step sizes should be one or
two orders of magnitude smaller than their corresponding initial
value, although the step size will be adjusted dynamically by the DEMC
algorithm. Note that I have set the step size on t34, the egress time,
to -4. This will fix the egress time to the value of the parameter in
the 4th position, the ingress time. Unless your planet's orbit is
extremely eccentric, it is probably a good idea to set these
parameters equal.

Note that a step size of 0 will fix a parameter to the initial value; it
will not be varied in the MCMC. This can be useful if the data set does
not constrain a value well, and any Gaussian prior you can put on the model
is too wide to make any impact.

Lastly, you you need to edit the ``p6'' section of your renamed
eventname.pcf in the ''run'' directory. The options in p6.pcf are as
follows:

\begin{itemize}
\item centering -- 'all' or a list of exact names of the centering
  directories. If the first item in the list is 'all', p6 will try to
  run on all subdirectories of the run directory that are named like a
  centering directory. Otherwise, p6 will ignore all centering
  directories not in the list.

\item photometry -- 'all' or a list of photometry directories. If
  'all', p6 runs on all photometry directories in the specified
  centering directories. If a list is given, p6 run on any photometry
  directory that matches a name in your list that is also in a
  centering directory specified by the previous option(centering).

\item modeldir -- the name of the output directory.

\item threads -- The number of parallel threads to run. Note that if
  mcmc is set to True, each thread will use nchains processes, so your
  total number of processes is nchains $\times$ threads.  If you are
  running MCMC, it is recommended to keep the thread count low.
\end{itemize}

It will save time to run p6 on a single combination of centering and
photometry, to find a good fit before trying to fit to all
combinations of centering and photometry. So edit p6.pcf to only
include a single centering method (e.g., fgc) and a single photometry
aperture (e.g., ap2500715).

Once you have configured these three files, you are ready to run p6. Use
the following command:

\begin{verbatim}
poet.py p6
\end{verbatim}

To have a date added to the output directory, do:

\begin{verbatim}
poet.py p6 nodate=False
\end{verbatim}

Alternatively, you can override p6.pcf by specifying a directory and outdir:

\begin{verbatim}
poet.py p6 modeldir=OUTDIR CETERDIR/PHOTDIR
\end{verbatim}

Or, to just override p6.pcf's outdir:

\begin{verbatim}
poet.py p6 modeldir=OUTDIR
\end{verbatim}

As p6 runs, it will display the MCMC's progress (unless MCMC was
toggled off). There are several things you should notice:

\begin{itemize}
\item Are the chains trying to leave their imposed boundaries? This would
  be evident in the list of number of times each parameter tried to step
  outside it's boundaries. If these numbers are the same order of magnitude
  of your iterations, then your boundaries may be too tight. Similarly,
  if the acceptance rate is very low (say less than 10\%), then your
  boundaries might be too tight.

\item Are the chains converging? Every 10\% of the iterations, POET
  will print the Gelman \& Rubin (GR) criterion for each parameter. If
  the chains are converging, these numbers will get closer to 1 as the
  chains continue. We say that model parameters converge if their GR
  criterion is within 1\% of unity (i.e., $<$1.01). If they are not
  converging, you may have unconstrained parameters, or you may have
  boundary issues.

\item Did the least-squares fit find a reasonable fit? Take a look at the
  best-fit parameters prior to the MCMC. Do they look reasonable for an
  eclipse, or are they pushed up against the imposed boundaries? If so,
  consider changing the boundaries or placing priors on parameters.

\end{itemize}

Once the run finishes, it will produce many plots, most located in a
subdirectory of the aperture directory. See Section \ref{sec:p6} for a
discussion of the plots and how to interpret them. If you make any
changes, rerun p6 and observe how the output changes.

Once things look good with the current set of models, try some other
ramp models. Compare the Bayesian Information Criterion (BIC), found
in the printout file (usually results.txt) between the results with
different ramp models. The BIC is related to chi-squared and is a
measure of goodness of fit, but it penalizes models with additional
parameters.  Find the ramp model which results in the lowest BIC. This
is your best model for whichever centering and photometry combination
you are running on.

Now we want to compare different centering and photometry
combinations. Go back to your initial run directory. For example:

\begin{verbatim}
cd /home/esp01/events/.../wa012bs12/
\end{verbatim}

Now we want to compare different centering and photometry
combinations. To do this, edit the p6 section of the renamed eventname.pcf
to now include all centering and photometry combinations you are
interested in. Then change to the top level directory and run p6:

\begin{verbatim}
cd ..
poet.py p6
\end{verbatim}

This will run all combinations of centerdir and photdir that you
specified, in series, which may take some time. If all goes well,
several plots will be placed in the plots directory (plots/OUTDIR
within your ``run'' directory).  Take a look at those labelled
sdnr*.png and depth*.png. They show standard deviation of normalized
residuals (SDNR) vs.\ aperture size for each centering method and
eclipse depth vs.\ aperture size for each centering method,
respectively. There is also a plot of the binned-$\sigma$ $\chi^2$
vs.\ aperture size, which is a measurement of correlated noise from
Deming et al. (2015). The final line output by p6 should tell you
which centering and photometry resulted in the best binned-$\sigma$
$\chi^2$.

Take a look at the depth vs.\ aperture size plot. Ideally, the eclipse
depth should not change drastically with aperture size. If it does
change significantly (i.e. it changes by more than 1$\sigma$), you may
need to do some investigation as to the cause, because a reviewer will
surely ask.

Now that you have determined the best centering and photometry, you
can go back to running p6 on a single centering and photometry
combination.  Rerun p6 with all the ramp models you tried earlier to
verify that the best ramp model has not changed with this change in
photometry and centering. If it has, check the difference in BIC
between the two models.  Is it very small? Then the choice of ramp
model is likely inconsequential, but you should compare model
parameter results to be sure they do not change significantly.

Assuming all has gone well up to this point, you are ready for a final
p6 run. Set the iterations to somewhere around 1e4 or 1e5 (the goal is
enough iterations to converge), make sure chi2flag is True, turn on
mcmc if it's toggled off, and run p6 one last time. Make sure that the
results have not changed too much with the MCMC.

If you are satisfied with the results, you can run p7. There is no
config file for p7, since the only input in the numcalc variable in
the params file from p6. To run p7, do:

\begin{verbatim}
poet.py p7 CENTERDIR/PHOTDIR/MODELDIR
\end{verbatim}

Replace MODELDIR with the directory name of the location of your p6 output
for the final run. p7 will append a number of results to your printout file
(normally results.txt), most importantly a brightness temperature
calculation and eclipst/transit times in BJD..

From here, you can simply run p8 and p9 the same way:

\begin{verbatim}
poet.py p8 CENTERDIR/PHOTDIR/MODELDIR
poet.py p9 CENTERDIR/PHOTDIR/MODELDIR
\end{verbatim}

p8 will add a nice table to your printout file, and p9 will produce some
fancy plots.

There is one final step -- making light curve tables with p10. Every
paper we submit comes with the actual data (the light curves) that we
analyzed. First, edit the p10.pcf file. This file is very particular
with formatting. Comments are allowed, but they must be on separate
lines and the first character of the line must be a \# character.
Lines are read individually as single entries, which allows for spaces.
There must be a blank line between each list, so that the file reader
knows when it reaches the end of a list. The `end' on the final line
of the file is necessary, so the reader knows when it reaches
the end of the file.

One you have edited p10.pcf to your liking, run p10 with:

\begin{verbatim}
poet.py p10 CENTERDIR/PHOTDIR/MODELDIR
\end{verbatim}

You will see a new directory irsa/ within your MODELDIR. This
directory will contain an IRSA table and FITS file for each
p6 output in MODELDIR. Note that it will only contain the fit
to the first set of models in your p6 run. This should not be a
problem, so long as your final p6 run only included the best
model.

At the time of this writing, p11 and p12 are executed on their own,
like:

\begin{verbatim}
lib/p11advtables.py RUNDIR/CENTERDIR/PHOTDIR/MODELDIR
lib/p12advfigs.py RUNDIR/CENTERDIR/PHOTDIR/MODELDIR
\end{verbatim}

They require some edits from the user, and are only useful for making
plots for publication. They are entirely optional.

Now you'll need to run ZEN, a different way to do light-curve
modeling.  ZEN covers everything from p6 through p12. At a high
level, ZEN applies the Pixel-Level Decorrelation (PLD) method,
as described in Deming et al. (2015). This method uses the values
of pixels themselves to remove the intrapixel effect, rather than
modeling intrapixel gain variability as is done in p6. For a full
discussion, see the paper by Deming et al. (2015).

Before you run ZEN, you will need to edit eventname-zen.cfg in the ``run''
directory. First, rename it (or create it):

\begin{verbatim}
cd wa012bs12
mv eventname-zen.cfg wa012bs12-zen.cfg
\end{verbatim}

The following describes the options in the configuration, which are
split into [EVENT] and [MCMC] sections:

\begin{itemize}
\item [EVENT]
  \begin{itemize}
\item eventname -- This is the event code for your observation. E.g.,
  wa012bs12. It must match the name of the p1 and p5 output files
  (minus the suffix).

\item modelfile -- The name of the file containing MCMC parameter values,
  boundaries, and step sizes.

\item outdir -- The name of the output directory that will be
  created. It should not already exist.

\item cent -- Centering directories to consider when optimizing
  centering methods, photometry apertures, and bin sizes.

\item phot -- Photometry directories to consider when optimizing
  centering methods, photometyr apertures, and bin sizes. The code
  will look for each ``phot'' directory within each ``cent''
  directory, skipping any combinations that do not exist.

\item bintry -- A list of bin sizes to try when optimizing centering
  methods, photometry apertures, and bin sizes. Setting this 0 will set
  to the default bin sizes, which is 1, 2, 4, 8, 16, etc. up to the
  maximum bin size.

\item maxbinsize -- The maximum bin size, non-inclusive. If this is
  set too high, the code will exit and instruct the user to decrease
  maxbinsize below a threshold.

\item nprocbin -- Number of CPUs used in the optimization.

\item nbinplot -- Number of bins in light curve plots.

\item preclip -- All data prior to this orbital phase will be clipped
  prior to modeling.

\item postclip -- All data after this orbital phase will be clipped
  prior to modeling.

\item interclip -- All data in specified orbital phase ranges will be
  clipped prior to modeling.

\item numcalc -- Number of calculations in the temperature Monte
  Carlo.  If this is higher than the number of MCMC iterations, it
  will be reduced to match the MCMC iterations.

\item models -- List of models included in the full model. Multiple
  model sets can be supplied, separated by newlines.

\item npix -- The number of pixels used in the PLD model. The default is
  9. If this is changed, the modelfile must be updated to reflect the
  new number of parameters in the PLD model.

\item priorvars -- A list of strings that match parameter names in
  python\_models.py.

\item priorvals -- A list of Gaussian prior means, lower uncertainties, and
  upper uncertainties for each parameter in ``priorvars''. There should
  be 3 numbers for each parameter in ``priorvars'', with the first being
  the mean, the second being the lower uncertainty, and the third
  being the upper uncertainty.

\item slopethresh -- A slope threshold for selection against correlated
  noise. If the slope of a line fit to the model residuals vs.\ bin size
  is higher than this threshold, that centering and photometry combination
  is rejected. In almost all cases, this number can be set to -0.3. Lower
  numbers are more strict, down to -0.5 (no correlated noise). If none
  of the selected centering and photometry combinations are below this
  threshold, the code will ignore the threshold and warn the user.

\item postanal -- True/False to complete post-analysis calculations.
  If False, the code will not calculate temperatures, eclipse/transit
  times, or generate light-curve tables, which can save significant
  time.

\item papername -- Name of your publication, if you intend to write
  one. This will be used in the topstring of the FITS and IRSA files
  that are generated.

\item month -- Month of paper submission.

\item year -- Year of paper submission.

\item journal -- Journal where paper will be submitted.

\item instruments -- Instruments used in observations.

\item programs -- Space-separated list of telescope program numbers.

\item authors -- Newline-separated list of paper authors.
\end{itemize}
\item [MCMC]
\begin{itemize}
\item nsamples -- The number of samples in the MCMC.

\item nchains -- The number of chains in the MCMC.

\item walk -- The MCMC walk method. See p6 description.

\item grtest -- A boolean flag telling the code to do the
  Gelman-Rubin convergence test. Should be true in most cases.

\item burnin -- The number of iterations which are removed from the
  posterior distributions, for each chain. Around 10\% of the total
  iterations per chain is sufficient.

\item plots -- Boolean for plotting. Toggling to False can save some
  run time.

\item bins -- Boolean for binning. Toggling to False will save
  considerable run time, but will disable bin size optimization.

\item savefile -- Name of the output file which will contain all
  samples from the MCMC.

\item logfile -- Name of the file which will contain all MCMC
  text output.

\item leastsq -- Boolean for doing a least-squares minimization of
  parameters prior to MCMC. Leave this as True in most cases, unless
  you want a specific starting point for your chains.

\item titles -- Boolean for titles on plots. 

\item chisqscale -- Boolean to rescale data uncertainties such that
  reduced chi-squared is 1. This helps with overestimated error bars.
  Leave as True for Spitzer data.

\item thinning -- MCMC chains will be thinned by this factor for
  plotting purposes. Increasing this can make plots generate faster,
  and be more readable.

  \end{itemize}
\end{itemize}

Once you are satisfied with your configuration, you can run ZEN with
the following command:

\begin{verbatim}
poet.py zen
\end{verbatim}

which will seek out your configuration file, provided it ends in ``-zen.cfg''. You can give a specific configuration file with:

\begin{verbatim}
poet.py zen PATH/TO/CONFIG.cfg
\end{verbatim}

This will run on all directories and find the best centering,
photometry, and binning. Afterwards, it creates an output directory in
the directory corresponding to the best centering and photometry named
with the date and the event code
(e.g. wa012bs12/fgc/ap2000715/zoutput-model01).  Figures can be
interpreted in the same way as p6 figures, in almost all
cases. IRSA/FITS files can be found in the irsa/ subdirectory within
the output directory.

Congratulations! You've done a complete analysis!

\section{Joint Fits}
POET is capable of fitting simultaneously to multiple light curves,
both with BLISS and PLD models.  This can be very useful if
you expect parameters to be the same between data sets. For example,
the midpoint and duration of a planet's eclipses should not change
appreciably between different observations, provided the observations
are not separated by a lot of time.

To do a joint fit, make a new ``run'' directory called joint, and some
dummy directories for the centering and photometry so that everything
is organized as it would be with a normal p6 run.

\begin{verbatim}
mkdir -p joint/fgc/ap250+000
\end{verbatim}

Now, copy the p5 output files from the runs which you want to
fit jointly. E.g. do something like this:

\begin{verbatim}
cp wa012bs11/fgc/ap250+000/wa012bs11_p5c.dat joint/cent/phot
cp wa012bs12/fgc/ap250+000/wa012bs12_p5c.dat joint/cent/phot
\end{verbatim}

If you want to do a joint fit with ZEN, you will also need to copy
the p1 output files from the observations you want to fit jointly.
For example:

\begin{verbatim}
cp wa012bs11/wa012bs11_ini.dat joint/
cp wa012bs11/wa012bs11_ini.h5  joint/
cp wa012bs12/wa012bs12_ini.dat joint/
cp wa012bs12/wa012bs12_ini.hg  joint/
\end{verbatim}

Then copy or create the required files for the joint fit, placing all
of them within the run directory (here named ``joint''). You will need

\begin{itemize}
\item For p6 joint fits:
  \begin{itemize}
\item A pcf with a params section for each data set to be fit jointly.

\item A single pcf with a p6 section which gives the location of the
  data to be fit (here centering = cent and photometry = phot). The
  name of this file must match the ``eventcode'' in poet.pcf.

\item An initvals.txt file for each data set, with names matching
  those specified in the params sections of the pcfs.
  \end{itemize}
\item For ZEN joint fits:
  \begin{itemize}
  \item A configuration file for each data set to be fit jointly, all with
    names ending in -zen.cfg.

  \item An initvals.txt file for each data set, with names matching those
    specified in the ZEN configuration files.
  \end{itemize}
\end{itemize}


If this is all set up correctly, you can simply run POET with

\begin{verbatim}
poet.py p6
\end{verbatim}

\noindent
or

\begin{verbatim}
poet.py zen
\end{verbatim}

For BLISS, the code will use the [p6] section of the first file it
reads, and the [params] sections of the pcf corresponding to each
dataset. For MCMC settings that cannot vary on a per dataset basis,
the code will use the option in the first file it reads in
alphanumeric order. It's best to set your MCMC options the same in all
the pcf files to prevent confusion.

For ZEN, the code will use the [EVENT] sections of each configuration
file it finds, and will use the [MCMC] section of the first file it
finds, in alphanumeric order. Again, it's best to set your MCMC
options the same in all the configuration files to prevent confusion.
You must also set the ``bintry'' value to a single number in all ZEN
configuration files, as bin size optimization is not supported during
joint fits (the code will exit). In most cases, you should use the
optimal bin sizes you determined from the fits to the individual data
sets.

To share parameter values between datasets, set the step size to a
negative number equal to the index of the parameter, just like
usual. Note, however, that the model parameters are concatenated into
a single list. For example, if you wanted to set the midpoint of the
mandelecl model in the second data set equal to the midpoint of the
mandelecl model in the first data set, then you would set the step
size of the midpoint in the second data set to -1. However, if you
wanted to share parameters between the second and third data sets, you
will need to count the total number of parameters in the models of the
first data set and offset accordingly.

\subsection{EDGAR}
\label{sec:POE}
EDGAR stands for Easy Distributed Global Aggregator of Runs. EDGAR can
be used to lazily run POET and do much of the analysis with minimal
intervention. 

In order to run EDGAR, you will need to have filled out the 'event',
'centering', 'photometry', and 'params' sections of the eventname.pcf, the
p10.pcf, and initvals.txt, all in the run directory. Now we will fill
out poet.pcf in the root directory (above the run directory). There
are two sections, POET:
\begin{itemize}
\item rundir -- the name of the directory to run EDGAR on, usually the
  same as eventname
\item eventname -- the event code of the observation, usually the same
  as rundir.
\end{itemize}
and EDGAR which is used exclusively by EDGAR:
\begin{itemize}
\item memory -- the amount of memory to allow EDGAR to use in MB
\item cores -- the maximum number of cores for EDGAR to use
\item test\_centering -- centering directory to test models on
\item test\_photometry -- photometry directory to test models on
\item model -- this parameter is interpreted as a two dimensional
  array where each line is a set of models. Each line after model
  (which includes sets of models) must be indented at least one space.
\item runp6 -- True/False, whether or not to run p6
\item runp7\_10 -- True/False, whether or not to run p7 through p10
\item runzen -- True/False, whether or not to run ZEN
\item zenmodels -- list of model set to run zen on
\item zenchains -- number of chains to use when running ZEN
\item zenmaxbinsize -- maximum bin size used in ZEN. Ignore unless zen
  spits out a error about maximum bin sizes. If it does, set this
  value according to that error output.
\end{itemize}

EDGAR will run p1-5 on all the centering and photometry specified in
their respcetive sections of eventname.pcf. The After running all the
aperture and photometry, p6 will be run multiple times in different
phases:

In every phase excpet the final phase (init, full, check), 'mcmc' and
'chi2scale' are disabled. In the final phase, these are enabled. These
two settings as well as 'model' are the only setting overwritten in
the 'params' section of eventname.pcf by EDGAR.

In 'init' phase, the specified models are compared on the
test\_centering and test\_photometry specified in poet.pcf. In 'full'
phase, the model found in 'init' phase is run on all
centering/photometry to find the best centering/photometry. In the
'check' phase, all models are rerun on the centering/photometry chosen
in the 'full' phase to ensure that a different model does not fit
better. In 'final' phase, the best centering/photometry/model is rerun
with a larger number of iterations. The best
centering/photometry/model will be printed out in the end.

After p6 is run in its various phases, p7 through p10 will be
run. After this, if `runzen` is ``True'', ZEN is run on all specified
`zenmodels`. Currently, the items listed in zenmodels are strings
coresponding to zen configuration files. There are a few default model
sets (noramp, linramp, quadramp), however, if you would like to run a
particular custom model set, it must be named ``zen\_<model>.cfg'' in
your run directory where ``<model>'' is one of the values you specify
for `zenmodels` poet.pcf.

NOTE: running ZEN will change configuration files, leaving
configuration files that reflect what was actually run. Only the
event, centering, and photometry sections of eventname.pcf,
zen\_<model>.cfg, poet.pcf, initvals.txt, and p10.pcf affect an EDGAR
run. To run EDGAR:

\begin{verbatim}
poet.py ed
\end{verbatim}

\section{Diagnosing by Loading the Event Object}
\subsection{p1 through p5}
At some point you will inevitably run into trouble, either with the
code or something specific to your data set. Fortunately, you can load
the event object, which contains all useful variables, into a python
session and examine it to look for problems. First, navigate to your
/run/ directory. Then launch a python 3 session

\begin{verbatim}
ipython3
\end{verbatim}

Once you have an interactive session going, we need to import some POET
codes to load the event object. Copy and paste the following code
into your session:

\begin{verbatim}
import sys
sys.path.append('lib/')
import manageevent as me
\end{verbatim}

You now have the manageevent module accessible in your session. At
this point, the command you use will differ, depending on which file
you wish to load. Each module of POET (p1, p2, etc.) produces an
output file which contains the event object. They are named according
to your event name and the module which produced them.

To load p1, assuming your ``run'' directory is wa012bs12 (adjust accordingly for other ``run'' directories)

\begin{verbatim}
event = me.loadevent('wa012bs12/wa012bs12_ini')
\end{verbatim}

To load p2:

\begin{verbatim}
event = me.loadevent('wa012bs12/wa012bs12_bpm')
\end{verbatim}

To load p3, assuming you used fgc for your centering method (adjust accordingly
for other methods):

\begin{verbatim}
event = me.loadevent('wa012bs12/fgc/wa012bs12_ctr')
\end{verbatim}

To load p4, assuming you used fgc for you centering method and a constant
aperture radius of 2.5 with sky annulus radii of 7 and 15 pixels (adjust
accordingly to your run):

\begin{verbatim}
event = me.loadevent('wa012bs12/fgc/ap250+000/wa012bs12_pht')
\end{verbatim}

To load p5, with the same assumptions as above (adjust according to your run):

\begin{verbatim}
event = me.loadevent('wa012bs12/fgc/ap250+000/wa012bs12_p5c')
\end{verbatim}

Note that me.loadevent does not, by default, load the entire data array
for memory concerns. If you need the data, uncertainties, or full
data mask, you can set these in a keyword argument, like so:

\begin{verbatim}
event = me.loadevent('wa012bs12/fgc/ap250+000/wa012bs12_p5c', load=['data','uncd','mask'])
\end{verbatim}

\subsection{p6 and p7}
p6 and p7 use a different function to load the event object because
they have added a fit object. For p6, from the top directory
(containing poet.py) do the following, replacing MODELDIR with the
location of the p6 output:

\begin{verbatim}
import sys
sys.path.append('lib')
import run
events = run.p6Restore('wa012bs12/fgc/ap2500715/MODELDIR')
\end{verbatim}

Loading the p7 output is very similar. From the same location, do the following
(again replacing MODELDIR):

\begin{verbatim}
events = run.p7Restore('wa012bs12/fgc/ap2500715/MODELDIR')
\end{verbatim}

Note that these functions, unlike me.loadevent, return a list of event
objects, even if you just loaded one.

\subsection{ZEN}
ZEN uses the same functions as p6 when saving and loading the event object.

\subsection{Interpreting the Event Object}
You now have a variable 'event' (or a list, 'events') which contains
all the products of POET. Some useful variables:

\begin{itemize}
\item event.fp -- This object contains the frame parameters (hence
  'fp'). Every value that is calculated or supplied on a
  frame-by-frame basis should be found here. Examples include:
  event.fp.x and event.fp.y, which are the centering positions of the
  target; event.fp.aplev, the raw photometry; and event.fp.aperr, the
  photometry uncertainty.

\item event.fit -- This object is a list contains the results of all model fits.
  Note that even if you only fit one model, event.fit will still be a list.

\item event.phase -- Time of each frame, converted to units of orbital phase.

\item event.params -- Object containing all params set in p6.
\end{itemize}

For example, if I wanted to plot the position of the target as a function
of time, I might do the following:

\begin{verbatim}
import sys
import matplotlib.pyplot as plt
sys.path.append('lib/')
import manageevent as me
event = me.loadevent('wa012bs12/fgc/ap250+000/wa012bs12_p5c')
plt.plot(event.phase, event.fp.x[0])
plt.show()
\end{verbatim}

Occasionally you may want to take a closer look at a POET plot. Most
of them can be recreated in an interactive session as long as you
have the event object loaded and know where the function that generated
the plot is located. For example, the p6 plots are made by function
in lib/plots.py. If you wanted to recreate the normalized light curve
plot, you might do something like the following:

\begin{verbatim}
import sys
import matplotlib.pyplot as plt
sys.path.append('lib')
import run
event = run.p6Restore('wa012bs12/fgc/ap250+000/p6output/')
import plots
plots.normlc(event[0], event[0].fit[0], 1)
plt.show()
\end{verbatim}

Note that you could make any changes to the plot that you wanted
between the calls to normlc() and plt.show().

The following is a far-from-complete list of useful attributes. If the
attribute does not exist, it is probably created by a module that you
have not run yet. There are hundreds of variables contained within the
event object. If you find some useful that you think may benefit
future users of POET, I encourage you to add them here.

\begin{itemize}
\item event.fp -- Frame Parameters object. Most parameters that are defined
  on a frame-by-frame basis are stored here. Shapes of contained arrays
  are generally npos by maxnimpos, where maxnimpos is the maximum number
  of frames taken at any particular image.
  \begin{itemize}
  \item event.fp.aplev -- Flux per frame.
  \item event.fp.apraw -- Flux per frame, without background subtraction.
  \item event.fp.x -- X position of the target on the detector.
  \item event.fp.y -- Y position of the target on the detector.
  \item event.fp.noisepix -- Noise pixels per frame.
  \item event.phase, event.fp.phase -- Time per frame in units of orbital phase.
  \item event.bjdtdb, event.fp.bjdtdb -- Time per frame in BJD$_{\textrm{TDB}}$
  \end{itemize}
\item event.fit -- A fit object, or list of fit objects, which contain
  the results of p6 or ZEN. Every set of models gets a separate fit
  object within the corresponding event object. For the following
  attributes, I will assume that event.fit is a single instance of a
  fit object. In real applications, you will need to index to get the
  fit object you want (in most cases, replace fit with fit[0]).
  \begin{itemize}
  \item event.fit.bestp -- Best-fitting parameters.
  \item event.fit.normsigma -- Clipped normalized uncertainties.
  \item event.fit.normsigmauc -- Unclipped normalized uncertainties. This
    pattern of adding 'uc' to variable names to get the unclipped version
    applies in many cases.
  \item event.fit.normbinflux -- Binned, clipped, normalized flux.
  \item event.fit.normbinstd -- Binned, clipped, normalized flux uncertainty.
  \item event.fit.normbestfit -- Normalized best-fit model.
    \item event.fit.binbjdtdb -- Binned BJD$_{\textrm{TDB}}$.
  \item event.fit.residuals -- Model fit residuals.
  \item event.fit.bestip -- Best-fit intrapixel model.
  \end{itemize}
\end{itemize}
  

\section{Testing POET}
This section gives the commands to run POET through its entirety,
for testing when making changes to the code.

Clone the POET repository:

\begin{verbatim}
git clone --recursive /home/esp01/git/poet.git POET/
\end{verbatim}

Rename the run directory to easily keep track of output

\begin{verbatim}
cd POET
mv run wa012bs12
\end{verbatim}

Build various c modules, documents, submodules, etc.

\begin{verbatim}
make
\end{verbatim}

Copy the necessary inputs from the test directory:

\begin{verbatim}
cp /home/esp01/events/POET-test/wa012bs12/*.pcf  wa012bs12/
\end{verbatim}

Set the proper run directory in poet.pcf

\begin{verbatim}
cp /home/esp01/events/POET-test/poet-wa012bs12.pcf poet.pcf
\end{verbatim}

Run p1 - p5. Note that this will use ~5 cores (can be adjusted in
wa012bs12.pcf). Run this before testing EDGAR as EDGAR will change
configuration files around.

\begin{verbatim}
./poet.py p1
\end{verbatim}

Copy initval file from the test directory

\begin{verbatim}
cp /home/esp01/events/POET-test/wa012bs12/*-initvals.txt wa012bs12/
\end{verbatim}

Copy ZEN config file from test directory

\begin{verbatim}
cp /home/esp01/events/POET-test/wa012bs12/*.cfg wa012bs12/
\end{verbatim}

Run p6

\begin{verbatim}
poet.py p6 outdir=p6output
\end{verbatim}

This will create a new directory in each photometry directory yo
specified in the p6 section of your eventname.pcf file with your p6
output, named p6output. Replace P6DIR with this directory, plus the
cent/phot directory. E.g. fgc/ap2500715/p6output

Run p7 - p9 and p12:

\begin{verbatim}
poet.py p7  <P6DIR>
poet.py p8  <P6DIR>
poet.py p9  <P6DIR>
poet.py p10 <p6DIR>
\end{verbatim}

Run zen

\begin{verbatim}
poet.py zen wa012bs12/zen.cfg
\end{verbatim}

Now repeat the above for a different WASP-12b observation so that
we can do a joint fit:

\begin{verbatim}
mkdir wa012bs22
cp /home/esp01/events/POET-test/wa012bs22/*.pcf wa012bs22/
cp /home/esp01/events/POET-test/poet-wa012bs22.pcf poet.pcf
./poet.py p1
cp /home/esp01/events/POET-test/wa012bs22/*-initvals.txt wa012bs22/
cp /home/esp01/events/POET-test/wa012bs22/*.cfg wa012bs22/
poet.py p6
poet.py p7  <P6DIR>
poet.py p8  <P6DIR>
poet.py p9  <P6DIR>
poet.py p10 <p6DIR>
poet.py zen wa012bs22/zen.cfg
\end{verbatim}

Make a directory for joint fits and copy the required files.

\begin{verbatim}
mkdir -p joint/fgc/ap2500715
cp wa012bs12/fgc/ap2500715/*.dat joint/fgc/ap2500715
cp wa012bs22/fgc/ap2500715/*.dat joint/fgc/ap2500715
cp /home/esp01/events/POET-test/joint/*.pcf joint/
cp /home/esp01/events/POET-test/joint/*-initvals.txt joint/
cp /home/esp01/events/POET-test/poet-joint.pcf poet.pcf
\end{verbatim}

Run the joint fit

\begin{verbatim}
poet.py p6
\end{verbatim}

Compare outputs to the results in /home/esp01/events/POET-test.

Now test EDGAR. EDGAR handles many aspects of the configuration files
for the user, and will replace those that were copied above. EDGAR,
for the most part, executes POET in the same way as above, but changes
to EDGAR should still be tested.

\begin{verbatim}
poet.py ed
\end{verbatim}

Compare outputs to the results in /home/esp01/events/POET-test.

\end{document}


