%\usepackage{margin}
\usepackage{graphicx}
\usepackage{enumitem}

\usepackage{amssymb, amsmath}
%\usepackage{fancyvrb}
%\usepackage{fixltx2e}

\usepackage[usenames,dvipsnames]{xcolor}
%\usepackage{xcolor}

\usepackage{listings}
%\usepackage{pxfonts}
%\usepackage[tiny,compact]{titlesec}
%\usepackage{bera}
%\usepackage{alltt}
%\renewcommand{\ttdefault}{txtt}

% To use boldface verbatim:
%\lstset{basicstyle=\ttfamily,
%        escapeinside={||},
%        mathescape=true}

\lstset{
    language={[LaTeX]TeX},
    basicstyle=\tt\color{red},
    escapeinside={||},
}

%\bibliographystyle{jupiter}
\bibliographystyle{aa}
%\bibliographystyle{apj}
%\bibliographystyle{icarus}

\def\bibAnnoteFile#1{}
\usepackage{natbib}
\bibpunct[, ]{(}{)}{,}{a}{}{,}
\usepackage{astjnlabbrev-jh}
\usepackage{bibentry}
\setlength\bibsep{0pt}
\usepackage{commath}
\usepackage{rotating}


% :::: jhmacs2.tex :::::

\typeout{Joe Harrington's personal setup, Wed Jun 17 10:53:17 EDT 1998}
% :::::: pato.tex ::::::
% Joetex character unreservations.
% This file frees most of TeX's reserved characters, and provides
% several alternatives for their functions.

% Tue Mar 29 22:23:03 EST 1994

% utility
\catcode`@=11

% comments are first....

\long\def\comment#1{}
\def\com{}
%\def\commenton{\catcode`\%=14}
%\def\commentoff{\catcode`\%=12}

\comment{$}
\let\mathshift=$
\def\mathstart{\begingroup $}
\def\mathstop{$ \endgroup}
\def\math#1{$#1$}
\def\mathshifton{\catcode`\$=3}
\def\mathshiftoff{\catcode`\$=12}

\def\greek#1{\math{#1}}

\let\oldbackslash=\backslash
\def\backslash{\ifmmode\oldbackslash\else$\oldbackslash$\fi}

\comment{alignment tab}
\let\atab=&
\def\atabon{\catcode`\&=4}
\def\ataboff{\catcode`\&=12}

\comment{super- and subscripts -- treat as a unit. \sp and \sb already
exist in plain TeX math mode, but not in regular text.  In fact,
superscripting and subscripting are automatic only in math mode (see
TeXbook p. 134.  Elsewhere, they either need to be simulated or faked
in math mode.  Use math mode or \msb and \msb when dealing with
numbers or they get too big (compare 10\sb{2} to \math{10\sb{2}}).}
\let\oldmsp=\sp
\let\oldmsb=\sb
\def\sp#1{\ifmmode
	   \oldmsp{#1}%
	 \else\strut\raise.85ex\hbox{\scriptsize #1}\fi}
\def\sb#1{\ifmmode
	   \oldmsb{#1}%
	 \else\strut\raise-.54ex\hbox{\scriptsize #1}\fi}
\newbox\@sp
\newbox\@sb
\def\sbp#1#2{\ifmmode%
	   \oldmsb{#1}\oldmsp{#2}%
	 \else
	   \setbox\@sb=\hbox{\sb{#1}}%
	   \setbox\@sp=\hbox{\sp{#2}}%
	   \rlap{\copy\@sb}\copy\@sp
	   \ifdim \wd\@sb >\wd\@sp
	     \hskip -\wd\@sp \hskip \wd\@sb
	   \fi
	\fi}
\def\msp#1{\ifmmode
	   \oldmsp{#1}
	 \else \math{\oldmsp{#1}}\fi}
\def\msb#1{\ifmmode
	   \oldmsb{#1}
	 \else \math{\oldmsb{#1}}\fi}
\def\supon{\catcode`\^=7}
\def\supoff{\catcode`\^=12}
\def\subon{\catcode`\_=8}
\def\suboff{\catcode`\_=12}
\def\supsubon{\supon \subon}
\def\supsuboff{\supoff \suboff}


\comment{active character -- seems pointless to have a function to
replace the single-character-to-replace-a-function.  If you want ~ as
an active character, reenable it.}
\def\actcharon{\catcode`\~=13}
\def\actcharoff{\catcode`\~=12}


\comment{parameter -- just turn on and off when needed.  Doing the
following:

\let\defplain=\def
\defplain\def{\paramon\afterassignment\paramoff\defplain}

barfs in \long\def.

Putting \paramon \paramoff in \begin{table} and \end{table} might be
good... amgreene mentionned something about \halign, too.

}

\def\paramon{\catcode`\#=6}
\def\paramoff{\catcode`\#=12}

\comment{brackets -- now these produce the Spanish inverted
punctuation.  \spanishoff will return them to their normal English
function of being angle brackets.}

\let\spanishexcl=<
\let\spanishques=>
\let\oldlt=<
\let\oldgt=>
\def\spanishexclon{\catcode`\<=12}
\def\spanishexcloff{\catcode`\<=13
		\def <{\ifmmode\oldlt\else$\oldlt$\fi}}
\def\spanishqueson{\catcode`\>=12}
\def\spanishquesoff{\catcode`\>=13
		\def >{\ifmmode\oldgt\else$\oldgt$\fi}}

\def\spanishon{\spanishexclon \spanishqueson}
\def\spanishoff{\spanishexcloff \spanishquesoff}

\comment{ now fix some things this will break}

\comment{ ref
\let\oldref=\ref
\def\ref#1{\reservedcharson\oldref{#1}\reservedcharsoff}
}

\comment{ bibliography }
\let\oldcite=\cite
\def\cite#1{\reservedcharson\oldcite{#1}\reservedcharsoff}

\let\oldbibliography=\bibliography
\def\bibliography#1{\reservedcharson\oldbibliography{#1}\reservedcharsoff}

\comment{ And now to turn us totally on and off... }

\def\reservedcharson{ \mathshifton   \actcharon   \paramon}
\def\reservedcharsoff{\mathshiftoff  \actcharoff  \paramoff}

\comment{ this doesn't seem to work right for definitions...}
\def\nojoe#1{\reservedcharson#1\reservedcharsoff}

\catcode`@=12
\reservedcharson

% ::::::::::::::::::::::

\comment{joe.tex sets up \comment and character reservation stuff.
  Remember to turn on reserved characters before including other
  peoples' files, and before defining anything.}

\reservedcharson

\def\atachar{\catcode`@=11}
\def\atnotachar{\catcode`@=12}

\atachar

\comment{My macros.}
\comment{technical stuff}

\let\oldpm=\pm
\def\pm{\ifmmode\oldpm\else\math{\oldpm}\fi}
\def\by{\ifmmode\times\else\math{\times}\fi}
\newcommand\lt{\ifmmode<\else\math{<}\fi}
\newcommand\gt{\ifmmode>\else\math{>}\fi}
\let\oldsim=\sim
\def\sim{\ifmmode\oldsim\else\math{\oldsim}\fi}
\let\oldapprox=\approx
\def\approx{\ifmmode\oldapprox\else\math{\oldapprox}\fi}

\def\ttt#1{10\sp{#1}}
\def\tttt#1{\by\ttt{#1}}
\def\bttt#1{10\sp{{\bfseries #1}}}
\def\btttt#1{{\by\bttt{#1}}}

\comment{\input{greek.tex}}
\DeclareSymbolFont{UPM}{U}{eur}{m}{n}
\DeclareMathSymbol{\umu}{0}{UPM}{"16}
\let\oldumu=\umu
\renewcommand\umu{\ifmmode\oldumu\else\math{\oldumu}\fi}
\newcommand\micro{\umu}
\comment{\font\greek = psyr
\def\micro{{\greek m}}}
\comment{\def\micron{\math{\mu}m}}
\def\micron{\micro m}
\let\microns \micron
\def\microsec{\micro s}

\def\angstrom{\AA}
\let\angstroms \angstrom

\def\degree{\ifmmode\sp\circ\else\math{\sp{\circ}}\fi}
\let\degrees \degree
\def\decdegree{\ifmmode\rlap{.}\sp{\circ}\else\rlap{.}\math{\sp{\circ}}\fi}
\def\arcmin{\ifmmode\sp{'}\else\math{\sp{'}}\fi}
\def\decarcmin{\ifmmode\rlap{.}\sp{'}\else\rlap{.}\math{\sp{'}}\fi}
\def\arcsec{\ifmmode\sp{''}\else\math{\sp{''}}\fi}
\def\decarcsec{\ifmmode\rlap{.}\sp{''}\else\rlap{.}\math{\sp{''}}\fi}

\def\prim{\ifmmode\sp{\prime}\else\math{\sp{prime}}\fi}

\def\h{\sp{h}}
\def\m{\sp{m}}
\def\s{\sp{s}}
\def\decs{\rlap{.}\sp{s}}

\def\smfrac#1#2{\math{#1\over#2}}

\def\bm#1{{\mbox{\boldmath\math{#1}}}}

\comment{writing}

\comment{
\def\ie{{\em i.e.,}\ }
\def\eg{{\em e.g.,}\ }
}
\def\etal{{\em et al.}}

\comment{for aligning tables (\i was a dotless i, \ii is for ``naive'')}
\let\oi=\i
\def\ii{\"\oi}
\newbox{\wdbox}
\def\c{\setbox\wdbox=\hbox{,}\hspace{\wd\wdbox}}
\def\i{\setbox\wdbox=\hbox{i}\hspace{\wd\wdbox}}
\def\n{\hspace{0.5em}}
\def\m{\hspace{1em}}
\def\d{\phantom{$-$}}

\comment{good for addresses}
\def\tablebox#1{\begin{tabular}[t]{@{}l@{}}#1\end{tabular}}

\comment{notes}
\comment{\font\tinytt = cmtt8}
\def\marnote#1{\marginpar{\raggedright\tiny\ttfamily\baselineskip=9pt #1}}
\def\herenote#1{{\bfseries #1}\typeout{======================> note on page \arabic{page} <====================}}
\def\fillin{\herenote{fill in}}
\def\fillref{\herenote{ref}}

\comment{Page setup like I like it.}

\comment{ % margin settings -- change over to jhmar?
% \input{simplemargins.sty}
% \comment{paper isn't really 8.5x11 inches}
% \setlength{\smpageheight}{7.96875in}
% \setlength{\smpageheight}{10.96875in}
% \setallmargins{3.0cm}
}

\comment{% paragraphs
% \setlength{\parskip}{\baselineskip}
% \setlength{\parindent}{0em}
}

\comment{figs/tables}
\setcounter{totalnumber}{400}

\def\nocaption{\refstepcounter\@captype \@dblarg{\@nocaption\@captype}}

\long\def\@nocaption#1[#2]#3{\addcontentsline{\csname
  ext@#1\endcsname}{#1}{\protect\numberline{\csname 
  the#1\endcsname}{\ignorespaces #2}}{\ignorespaces #3}}

\comment{Date and time format}
\def\today{\number\day\space \ifcase\month\or
  January\or February\or March\or April\or May\or June\or
  July\or August\or September\or October\or November\or December\fi
  \space\number\year}

\newcount\@timect
\newcount\@hourct
\newcount\@minct
\def\now{\@timect=\time \divide\@timect by 60
	 \@hourct=\@timect \multiply\@hourct by 60
	 \@minct=\time \advance\@minct by -\@hourct
	 \number\@timect:\ifnum \@minct < 10 0\fi\number\@minct}

\comment{optional file inclusion}
\def\inclopt#1#2{\if#2y#1,\fi}
\def\texfopt#1#2{\if#2y#1\else null\fi}
\def\psfopt#1#2{\if#2y#1\else/dev/null\fi}

\comment{page styling}
\def\clearblankdoublepage{\clearpage\if@twoside \ifodd\c@page\else
    {\pagestyle{empty}\hbox{}\newpage\if@twocolumn\hbox{}\newpage\fi}\fi\fi}

\def\threehfbox#1#2#3{\hbox to \textwidth{\com
	\rlap{#1}\com
	\rlap{\hbox to \textwidth{\hfil #2 \hfil}}\com
	{\hfil #3}}}

\atnotachar
\reservedcharsoff

% ::::::::::::::::::::::


\def\vs{{\em vs.}}
\def\p{\phantom{(0)}}

\setcounter{secnumdepth}{3}
\actcharon
\renewcommand{\textfraction}{0.1}
\comment{\paramon\def\herenote#1{}\paramoff}
\renewcommand{\thepage}{\arabic{page}}
\reservedcharson
\comment{Must have ONLY ONE of these...}
\newcommand\jhauth[1]{{#1}}
\newcommand\jhstud[1]{{#1}}
\comment{
\newcommand\jhauth[1]{{\bfseries #1}}
\newcommand\jhstud[1]{{\em #1}}
}
% :::::::::::: My Additions ::::::::::::::
\newcommand\Spitzer{{\em Spitzer}}
\newcommand\SST{{\em Spitzer Space Telescope}}
\newcommand\chisq{$\chi^2$}
\newcommand\PT{$P$--$T$}
\newcommand\itbf[1]{\textit{\textbf{#1}}}
\newcommand\bftt[1]{\texttt{\textbf{#1}}}
\newcommand\function[1]{\noindent\texttt{\begin{tabular}{@{}l@{}l}#1\end{tabular}}\newline}
\newcommand\bfv[1]{|\textbf{#1}|}
\newcommand\ttred[1]{\textcolor{red}{\ttfamily #1}}
\newcommand\ttblue[1]{\textcolor{blue}{\ttfamily #1}}
\definecolor{gray}{gray}{0.6}
\newcommand\tgray[1]{\textcolor{gray}{#1}}

\newcommand\der{{\rm d}}
\newcommand\tno{$\sp{-1}$}
\newcommand\tnt{$\sp{-2}$}
\newcommand*\Eval[3]{\left.#1\right\rvert_{#2}^{#3}}
\newcommand\findme[1]{\herenote{(FINDME: #1)}}
%:::::::::::::::::::::::::::::::::::::::::
% Next six lines adjust spacing above/below captions and Sections etc
% Adjust as needed
% Python style for highlighting
\DeclareFixedFont{\ttnm}{T1}{txtt}{m}{n}{9.8}  % for normal

\newcommand\plainstyle{\lstset{
language=Python,
basicstyle = \ttnm,
keywordstyle  = \ttnm,      %
emph        = {MyClass, __init__},     % Custom highlighting
emphstyle   = \ttnm\color{black},    % Custom  highlighting style
stringstyle = \color{black},         % Strings highlighting style
commentstyle=\color{black},         % Comment highlighting style
frame       = tb,                      % Any extra options here
showstringspaces = false
}}

% Plain environment:
\lstnewenvironment{plain}[1][]{\plainstyle\lstset{#1}}{}

\newcommand\plaininline[1]{{\plainstyle\lstinline!#1!}}

\comment{
% \setlength{\abovecaptionskip}{0pt}
% \setlength{\belowcaptionskip}{0pt}
% \setlength{\textfloatsep}{8pt}
% \titlespacing{\section}{0pt}{5pt}{*0}
% \titlespacing{\subsection}{0pt}{5pt}{*0}
% \titlespacing{\subsubsection}{0pt}{5pt}{*0}
}

\reservedcharsoff
\actcharon
\mathshifton
